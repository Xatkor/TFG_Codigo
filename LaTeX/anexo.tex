\chapter*{Anexo}

Este anexo pretende explicar los aspectos más destacados del código desarrollado en Python y dar una idea de como se han realizado las simulaciones.

\vspace{3mm}

El código se separa en 4 partes donde cada una tiene su función

\begin{itemize}
    \item obstacles.py
    \begin{itemize}
        \item Contiene los objetos que definen la partícula y pared.
    \end{itemize}
    \item physics.py
    \begin{itemize}
        \item Contiene las funciones que permiten encontrar el siguiente obstáculo con el que colisionará la partícula, además del momento en el que lo hará.
    \end{itemize}
    \item simulation.py
    \begin{itemize}
        \item Aquí es donde se simulan los billares con las condiciones dadas desde alto nivel. 
    \end{itemize}
    \item plotter.py
    \begin{itemize}
        \item Se encarga de esbozar las trayectorias de la partícula y paredes, además de reportar la velocidad tras cada iteración de manera gráfica.
    \end{itemize}
\end{itemize}

\section*{Obstacles.py}

Define dos clases, una para la partícula 

\begin{lstlisting}[breaklines]
    class Ball:
    """ A ball which is going to move in the billiard"""

        def __init__(self, pos, vel, radius=0) -> None:
            """
            Args:
                pos: Position of the ball
                vel: Velocity of the ball.
                radius: Radius pf the ball. Default: 0
            """
            self.pos = np.asarray(pos)
            self.velocity = np.asarray(vel)
            self.radius = radius
\end{lstlisting}

y otra para las paredes

\begin{lstlisting}[breaklines]
    class InfiniteWall:
    """ An infinite wall where balls can collide only from one side. """

        def __init__(self, start_point, end_point, velocity, side, relativistic=False, restitution=1) -> None:
            """
            Args:
                start_point: A point of the wall.
                end_point: A point of the wall.
                velocity: Velocity of the wall. If relativistic is True, velocity must be a fraction of the speed of light
                side: Position of the wall: left, right, top, bottom.
                relativistic: True if wall is relativistic. Default: False
                restitution: Restitution of the ball. Default: 1.
                        elastic collision -> restitution = 1
                        inelastic collision -> restitution <= 1
            """
            self.start_point = np.asarray(start_point)
            self.end_point = np.asarray(end_point)
            self.velocity = np.asarray(velocity)
            self.side = side
            self.relativistic = relativistic
            self.restitution = restitution
    \end{lstlisting}

\vspace{3mm}

Dentro de la clase \textit{InfiniteWall} se calcula la normal de la pared para que se tenga un correcto posicionamiento y una función que implementa las ecuaciones \ref{eq:velocity_change1} y \ref{eq:velocity_change2} 

\begin{lstlisting}[breaklines]
def relativistic_velocity(self, vel):
    """ Calculate the velocity of a ball after colliding with the wall when relativistic mode is enabled

    Args:
        vel: Velocity of the ball

    Returns:
        tuple: velocity of the ball after impact
    """

    if self.side == "left" or self.side == "right":
        # wall_velocity = self.velocity[0]
        # denominator = 1 - 2 * wall_velocity * vel[0] + wall_velocity * wall_velocity
        # velocity_x = (-vel[0] + 2 * wall_velocity - vel[0] * wall_velocity * wall_velocity) / denominator
        # velocity_y = (1 - wall_velocity * wall_velocity) * vel[1] / denominator
        wall_velocity = self.velocity[0]
        denominator = 1 - (1 + self.restitution) * wall_velocity * vel[
            0] + wall_velocity * wall_velocity * self.restitution
        velocity_x = (-vel[0] * self.restitution + (1 + self.restitution) * wall_velocity - vel[
            0] * wall_velocity * wall_velocity) / denominator
        velocity_y = (1 - wall_velocity * wall_velocity) * vel[1] / denominator
    else:
        wall_velocity = self.velocity[1]
        denominator = 1 - (1 + self.restitution) * wall_velocity * vel[
            1] + wall_velocity * wall_velocity * self.restitution
        velocity_y = (-vel[1] * self.restitution + (1 + self.restitution) * wall_velocity - vel[
            1] * wall_velocity * wall_velocity) / denominator
        velocity_x = (1 - wall_velocity * wall_velocity) * vel[0] / denominator

    return velocity_x, velocity_y
\end{lstlisting}

\section*{Physics.py}

La función principal de este módulo es la que encuentra el tiempo que tarda la partícula en colisionar con todos los muros, y junto con otra llamada \textit{calc\textunderscore next\textunderscore obstacle} (devuelve los tiempos en orden ascendente) determina la pared con la que colisionará la partícula. Aquellos casos donde no colisione nunca devuelve un tiempo infinito.

\begin{lstlisting}[breaklines]
def time_of_impact_with_wall(pos1, vel1, radius, posW, obstacle):
    """ Calculate when the ball collide with vertical or horizontal wall.

    Args:
        pos1: Position of the ball.
        vel1: Velocity of the ball.
        radius: Radius of the ball.
        posW: Position of the wall.
        obstacle: Obstacle

    Returns:
        Time of the impact, it is infinite if no impact.

    """

    _normal = obstacle._normal
    velW = obstacle.velocity
    side = obstacle.side

    if _normal[0] != 0:
        # Vertical wall
        if vel1[0] == velW[0]:
            # Same velocity, never collide
            return INF
        if side == "left":
            t = (radius + posW[0] - pos1[0]) / (vel1[0] - velW[0])
        else:
            t = (-radius + posW[0] - pos1[0]) / (vel1[0] - velW[0])
    else:
        # Horizontal wall
        if vel1[1] == velW[1]:
            # Same velocity, never collide
            return INF
        if side == "top":
            t = (-radius + posW[1] - pos1[1]) / (vel1[1] - velW[1])
        else:
            t = (radius + posW[1] - pos1[1]) / (vel1[1] - velW[1])

    if t < 1e-9:
        # they don't collide
        return INF
    else:
        return t
\end{lstlisting}

\section*{Simulation.py}

Se compone de casi 500 líneas de código donde lo más importante a resaltar sea el trozo donde obtiene la pared con la que colisiona, la comprobación de colisión con una esquina y la actualización de los valores de la partícula para la siguiente colisión.

\begin{lstlisting}[breaklines]

times_obstacles = calc_next_obstacle(ball.pos, ball.velocity, ball.radius, obstacles)

t = times_obstacles[0][0]

if t == self.INF:
    # print(f"No more collisions after {i} collisions")
    break

# Update properties of the ball
if times_obstacles[1][0] - times_obstacles[0][
    0] < 1e-9:  # Time difference is to close, it collides with a corner
    if times_obstacles[0][1].side == "top" or times_obstacles[0][1].side == "bottom":
        new_ball_velocity_Y = times_obstacles[0][1].update(vel1)[1]
        new_ball_velocity_X = times_obstacles[1][1].update(vel1)[0]
    else:
        new_ball_velocity_X = times_obstacles[0][1].update(vel1)[0]
        new_ball_velocity_Y = times_obstacles[1][1].update(vel1)[1]
    vel1 = (new_ball_velocity_X, new_ball_velocity_Y)
    pos_ball = ball.pos + np.multiply(ball.velocity, t)
    ball.pos = pos_ball
    ball.velocity = vel1
else:
    new_ball_velocity = times_obstacles[0][1].update(vel1)
    pos_ball = ball.pos + np.multiply(ball.velocity, t)
    vel1 = new_ball_velocity
    ball.pos = pos_ball
    ball.velocity = vel1
\end{lstlisting}

Además, este módulo realiza las funciones de guardado de cada posición y velocidad de la simulación, llamar al módulo para graficar y calcula el área en cada colisión.

\vspace{3mm}

Para realizar la simulación correctamente se crea un script donde se importen las librerías necesarias y se definan las propiedades de las paredes, el número de iteraciones y el número de partículas a simular.

\begin{lstlisting}[breaklines]
from Billiards.simulation import Simulation2D
import numpy as np

def main():
    # -------------------------------------
    Relativistic_mode = True
    Coef_restitution = 1
    # -------------------------------------

    # -------------------------------------
    # Distance of walls
    # -------------------------------------
    top_wall_distance = 12000
    bottom_wall_distance = 1
    left_wall_distance = 1
    right_wall_distance = 21000

    wall_distances = [top_wall_distance, bottom_wall_distance, left_wall_distance, right_wall_distance]

    # -------------------------------------
    # Velocities of walls
    # -------------------------------------
    top_wall_velocity = np.array([0, -0.02], dtype=np.float64)
    bottom_wall_velocity = np.array([0, 0.01], dtype=np.float64)
    left_wall_velocity = np.array([0.01, 0])
    right_wall_velocity = np.array([0.02, 0])

    wall_velocities = [top_wall_velocity, bottom_wall_velocity, left_wall_velocity, right_wall_velocity]

    # -------------------------------------
    # SIMULATION
    # -------------------------------------
    billiard = Simulation2D(wall_velocities, wall_distances, Relativistic_mode, Coef_restitution)

    # Number of collision and particles
    num_of_iterations = 10000
    nmax = 1000

    # Run simulations
    velocities, positions = billiard.evolve(num_of_iterations, nmax)

    # Plot velocities and path
    billiard.show_billiard(ball_velocities_average=velocities, ball_positions=positions)
    
    # Save mean velocity
    file_name = f"2D_DDArea.txt"
    billiard.save_results(velocities, file_name)

if __name__ == "__main__":
    main()
\end{lstlisting}

