\documentclass[11pt, spanish]{book}
\usepackage{MiEstilo}

\begin{document}

\chapter{Resultados numéricos}

Para diversos casos donde las paredes se mueven mover o no, se va realizar la simulación de un partícula en el interior de un billar rectangular. Para ello el procedimiento es usar el código creado \textcolor{red}{Referenciar código mio} para \( 10^3 \) partículas, no interactuantes entre sí. De este modo es similar a realizar una simulación de un gas ideal donde cada partícula inicia con una velocidad y posición distinta al resto dentro del mismo recinto. Con los datos obtenidos se puede obtener la velocidad media tras cada colisión y comprobar la existencia de la aceleración de Fermi.

\begin{figure}[H]
    \begin{subfigure}[b]{0.5\textwidth}
        \centering
        \includegraphics[scale=0.35]{images/Billiard_general.pdf}
        \caption{$w_i<0$}
        \label{fig:a}
    \end{subfigure}
    \hfill
    \begin{subfigure}[b]{0.5\textwidth}
        \centering
        \includegraphics[scale=0.35]{images/Billiards_generico_2.pdf}
        \caption{$w=0$}
        \label{fig:s}
    \end{subfigure}
    \caption{Trayectoria de una partícula. a) En un billar con paredes dirigiéndose hacia el centro. b) Paredes con velocidad nula}
\end{figure}


Los billares que interesan simular son aquellos que tienen al menos una de las paredes con velocidad no nula, ya que, es así cuando la velocidad de la partícula se puede ver modificada. Además para realizar una mejor comparación vamos a separar los casos donde la partícula se encuentra en un espacio de una dimensión y el de dos dimensiones, comparando ambas teorías en sus correspondiente dimensión, y por otro lado que ocurre cuando las colisiones no son totalmente elásticas.

\section{Caso unidimensional}

Nos situamos en un espacio donde las partículas están confinadas entre las paredes de un billar rectangular, y donde tiene su movimiento está restringido a la línea recta que une dos paredes paralelas. Con esta restricción y la posibilidad de que esas paredes paralelas puedan moverse sobre esa misma línea vamos a ver si los resultados numéricos afirman la teoría.

\vspace{3mm}

Con estas condiciones vamos a simular situaciones donde la distancia \( d(t) \) va a dejar de ser constante. De otro modo, la partícula oscilaría entre dos paredes moviéndose en la misma dirección y misma velocidad donde la partícula aumentaría y disminuiría su velocidad en un mismo factor y no habría interés en su estudio.

\subsection{Clasico}

Según la teoría desarrollada en el capítulo anterior cuando al menos una de las paredes se mueve de forma que su movimiento tiende a aumentar el área del billar, en este caso la distancia entre las paredes se ve aumentada, entonces la velocidad disminuye. Esto se ve también en las simulaciones (Figura \ref{fig:clasico_1D_A}) llegando además a una velocidad donde la partícula ve modificada su velocidad hasta una menor que la velocidad de la pared a la que se dirige.

\begin{figure}[H]
    \centering
    \includegraphics[scale=0.65]{images/Simulaciones/Clasico_1D/1D_A-N1000.pdf}
    \caption{Velocidades medias tras cada colisión en una dimensión. La distancia entre paredes aumenta}
    \label{fig:clasico_1D_A}
\end{figure}


\begin{figure}[H]
    \centering
    \includegraphics[scale=0.65]{images/Simulaciones/Clasico_1D/1D_B-N1000.pdf}
    \caption{Velocidades medias tras cada colisión en una dimensión. La distancia entre paredes disminuye}
    \label{fig:clasico_1D_B}
\end{figure}

Si nos situamos en el caso contrario, donde la distancia entre paredes se ve disminuida, vemos que los resultados son inversos (Figura \ref{fig:clasico_1D_B}). La velocidad tiende hacia el infinito sin aproximarse a un ningún valor al igual que se podía ver en las Cobwebs (Figura \ref{fig:Cobweb_Classic_1D} \textcolor{red}{En otro capitulo}). 

\vspace{3mm}

Es interesante ver también los casos donde las paredes tienen velocidades distintas donde una intenta ampliar la distancia y la otra acortarla. 

\begin{figure}[H]
    \begin{subfigure}[b]{0.5\textwidth}
        \centering
        \includegraphics[scale=0.55]{images/Simulaciones/Clasico_1D/1D_C1-N100.pdf}
        \caption{$d \rightarrow \infty$}
        \label{fig:clasico_distancia_infinito}
    \end{subfigure}
    \hfill
    \begin{subfigure}[b]{0.5\textwidth}
        \centering
        \includegraphics[scale=0.55]{images/Simulaciones/Clasico_1D/1D_C2-N1000.pdf}
        \caption{$d \rightarrow 0$}
        \label{fig:clasico_distancia_cero}
    \end{subfigure}
    \caption{Velocidades medias según la distancia entre paredes.}
\end{figure}

Se observa como el ampliar o reducir la distancia cambia totalmente el sistema. En el primer caso (Figura \ref{fig:clasico_distancia_cero}) las distancia entre las paredes aumenta lentamente y se ve como es necesario varias colisiones antes de conseguir una velocidad estable. Por el otro lado (Figura \ref{fig:clasico_distancia_infinito}), la distancia se reduce y su velocidad media aumenta tras cada colisión sin llegar a un valor estable, además de tender hacia el infinito. 

\vspace{3mm}

Con estas últimas simulaciones se puede afirmar que dado una billar con ambas paredes en movimiento donde las paredes se persiguen es posible encontrar otro billar donde únicamente una donde las paredes sea móvil y obtener los mismos resultados. En el caso que las paredes posean direcciones distintas también será posible encontrar un billar equivalente dado que la velocidad cambia tras cada colisión y sólo es necesario encontrar una velocidad para la pared que tras cada colisión aumente o disminuya la velocidad un factor que queramos. 

\subsection{Relativista}

Si nos situamos ahora en el contexto relativista sabemos que no es posible superar la velocidad para ninguna partícula, además de haberlo comprobado teóricamente en el capítulo anterior. Realizando la mismas simulaciónes pero con velocidades en función de la velocidad de la luz \( u \in [0, 0.5c] \), vemos como los resultados tienen una forma similar a la teoría clásica. Cuando se tiene un muro en reposo y otro con un movimiento con tendencia a aumentar la distancia, la velocidad media disminuye hasta un valor constante (Figura \ref{fig:relatividad_1D_A}). 

\vspace{3mm}

El caso opuesto donde las paredes se distancian (Figura \ref{fig:relatividad_1D_B}), se observa también un aumento de la velocidad pero aquí si vemos el límite que toma, el de la velocidad de la luz. Cuando la pared realiza un movimiento ``lento'' no se consigue alcanzar una velocidad casi a la de la luz hasta 3.000 colisiones después, en cambio con una velocidad mayor (10\% el de la luz) apenas se necesitan 20 colisiones. Vemos así la importancia que obtiene la velocidad de las paredes para obtener una velocidad elevada en poco tiempo.

\vspace{3mm}

\begin{figure}[H]
    \centering
    \includegraphics[scale=0.65]{images/Simulaciones/Relatividad_1D/1D_A-N1000.pdf}
    \caption{Velocidades medias tras cada colisión en una dimensión. La distancia entre paredes aumenta}
    \label{fig:relatividad_1D_A}
\end{figure}

\begin{figure}[H]
    \centering
    \includegraphics[scale=0.65]{images/Simulaciones/Relatividad_1D/1D_B-Juntos-N1000.pdf}
    \caption{Velocidades medias tras cada colisión en una dimensión. La distancia entre paredes aumenta}
    \label{fig:relatividad_1D_B}
\end{figure}

Se pueden observar distintos escenarios con ambas paredes móviles. En primer lugar (Figura \ref{fig:relatividad_1D_C}), la distancia entre las paredes aumenta, lo que resulta en una velocidad media descendente hasta que se alcanza una velocidad constante. Posteriormente, cuando la distancia entre las paredes disminuye (Figura \ref{fig:relatividad_1D_C2}), se observa nuevamente un aumento en la velocidad, aproximándose a la velocidad de la luz.

\vspace{3mm}

Se puede notar la diferencia que existe entre la rapidez a la que la velocidad crece y decrece. Cuando la distancia aumenta la velocidad pérdida en 500 colisiones es de un \( 60\% \) mientras que al disminuir la distancia en el mismo número de colisiones aumenta en un \( 586\% \). Esto hace ver la gran aceleración que se ve producida aún teniendo la tasa de cambio de la distancia en un mismo factor (las paredes interecambian velocidades de un caso a otro).

\begin{figure}[H]
    \centering
    \includegraphics[scale=0.65]{images/Simulaciones/Relatividad_1D/1D_C-N1000.pdf}
    \caption{Velocidades medias tras cada colisión en una dimensión. La distancia entre paredes aumenta}
    \label{fig:relatividad_1D_C}
\end{figure}

\begin{figure}[H]
    \centering
    \includegraphics[scale=0.65]{images/Simulaciones/Relatividad_1D/1D_C2-N1000.pdf}
    \caption{Velocidades medias tras cada colisión en una dimensión. La distancia entre paredes aumenta}
    \label{fig:relatividad_1D_C2}
\end{figure}

\subsection{Colisión inelástica}

Hasta ahora hemo visto como tanto para el caso clásico como para el relativista obtenemos una aceleración cuando la distancia entre paredes se ve aumentada. Por lo tanto estamos en presencia de una aceleración de Fermi. En el mundo real esta aceleración no podría darse dado que entre las sucesivas colisiones las partículas se ven sometidas a distintas fuerzas que reducen la velocidad (p.e. fuerza de fricción). 

\vspace{3mm}

De las distintas formas que pueden suprimir la aceleración vamos a utlizar la pérdida de energía que se produce en cada colisión, así estaríamos ante colisiones inelásticas. Recordemos que esta supresión sólo se aplica al caso relativista, ya que, es el que más interés causa al ser el más cercano a la realidad.

\vspace{3mm}

Se debe tener en cuenta que las soluciones obtenidas en el capítulo anterior suponen colisiones sucesivas con una pared con la misma velocidad y siempre en sentido contrario a la dirección de la partícula. Hasta ahora sólo se han visto billares donde una pared era móvil o tenían una velocidad cercana entre sí. Esto era así por la equivalencia que se podía encontrar entre billares para llegar a una velocidad muy cercana a la de la luz sin importar el número de colisiones que eran necesarias para llegar a ese valor. Estos casos estudiados teóricamente se pueden ver en las figuras \ref{fig:1D_inelastic_A} y \ref{fig:1D_inelastic_B}. Podemos notar como una colisión casi inelástica reduce drásticamente la velocidad máxima posible de la partícula para una misma velocidad dada.

\begin{figure}[H]
    \begin{subfigure}[b]{0.5\textwidth}
        \centering
        \includegraphics[scale=0.55]{images/Simulaciones/Inelastico_1D/1D_A-N1000.pdf}
        \caption{$\epsilon = 0.99$}
    \end{subfigure}
    \hfill
    \begin{subfigure}[b]{0.5\textwidth}
        \centering
        \includegraphics[scale=0.55]{images/Simulaciones/Inelastico_1D/1D_C-N1000.pdf}
        \caption{$\epsilon = 0.7$}
    \end{subfigure}
    \caption{Velocidades medias según la distancia entre paredes.}
    \label{fig:1D_inelastic_A}
\end{figure}

\begin{figure}[H]
    \begin{subfigure}[b]{0.5\textwidth}
        \centering
        \includegraphics[scale=0.55]{images/Simulaciones/Inelastico_1D/1D_B-N1000.pdf}
        \caption{$\epsilon = 0.99$}
    \end{subfigure}
    \hfill
    \begin{subfigure}[b]{0.5\textwidth}
        \centering
        \includegraphics[scale=0.55]{images/Simulaciones/Inelastico_1D/1D_D-N1000.pdf}
        \caption{$\epsilon = 0.7$}
    \end{subfigure}
    \caption{Velocidad media de las partículas tras colisiones inelásticas con paredes que se mueven hacia el interior del billar a la misma velocidad.}
    \label{fig:1D_inelastic_B}
\end{figure}

\subsection{Análisis de los resultados}

Como se ha podido ver las gráficas mostradas se asemejan a funciones contínuas. En los casos donde aparecía la aceleración de Fermi se pueden aproximar a funciones sigmoide

\begin{equation}
    f(n) = \dfrac{\beta}{\beta + e^{-n\alpha}}
\end{equation}

e incluso con mayor exactitud a una función de distribuión acumulativa

\begin{equation}\label{eq:acumulativa_general}
    f(n) = \delta - \gamma\exp(-\beta n^\alpha)
\end{equation}

donde \( \alpha, \beta, \gamma, \delta \) son constantes a determinar para cada billar. Estas constantes de alguna manera van a estar relacionadas con las propiedades del billar (velocidad y coeficiente de restitución). La ecuación \ref{eq:acumulativa_general} marca el parámetro \( \eta \) como el valor máximo que puede alcanzar esa función, entonces su valor va a limitar la velocidad de la partícula. Esa limitación depende fuertemente del tipo de colisión que se vaya a considerar y la velocidad que tengan las paredes, por lo tanto, ese valor \( \delta \) saldrá de la ecuación \ref{eq:Solucion_cobweb_inelastica}\textcolor{red}{En otro capitulo}. El parámetro \( \alpha \) determina como de horizontal es la función, valores \( c < 0 \) se curva hacia arriba, valores \( c > 0 \) se curva hacia abajo. El parámetro \( \beta \) determina que rápido se llega al valor máximo (tendrá relación con la velocidad de la pared.). El parámetro \( \alpha \) es similar a \( \beta \) en cuanto a dependencia pero no de valor.

\vspace{3mm}

\begin{figure}[H]
    \begin{subfigure}[b]{0.5\textwidth}
        \centering
        \includegraphics[scale=0.55]{images/Simulaciones/Relatividad_1D/1D_B-N1000_fit.pdf}
        \caption{$w_r = -0.001 \quad w_l = 0$}
    \end{subfigure}
    \hfill
    \begin{subfigure}[b]{0.5\textwidth}
        \centering
        \includegraphics[scale=0.55]{images/Simulaciones/Relatividad_1D/1D_B2-N1000_fit.pdf}
        \caption{$w_r = -0.01 \quad w_l = 0$}
    \end{subfigure}
    \hfill
    \begin{subfigure}[b]{0.5\textwidth}
        \centering
        \includegraphics[scale=0.55]{images/Simulaciones/Relatividad_1D/1D_B3-N1000_fit.pdf}
        \caption{$w_r = -0.1 \quad w_l = 0$}
    \end{subfigure}
    \hfill
    \begin{subfigure}[b]{0.5\textwidth}
        \centering
        \includegraphics[scale=0.55]{images/Simulaciones/Relatividad_1D/1D_C2-N1000_fit.pdf}
        \caption{$w_r = 0.01 \quad w_l = 0.011$}
    \end{subfigure}
    \caption{Ajuste de los datos a la ecuación \ref{eq:acumulativa_general}. Los valores \( w_l \) y \( w_r \) corresponden a la velocidad de la pared derecha e izquierda, respectivamente. Todas las colisiones son totalmente elásticas.}
    \label{fig:ajuste_1D_relativity}
\end{figure}

\vspace{3mm}

Se observa al ajustar los datos (Figura \ref{fig:ajuste_1D_relativity}) que todas las funciones se aproximan de manera muy exacta. Además el parámetro \( \gamma \) toma un valor constante \( \approx 0.74 \). Los valores que se mantienen constanten en estos billares es el coeficiente de restitución, la velocidad máxima que puede adquirir la partícula, la velocidad del muro, y la velocidad media de las partículas. Con estos valores nombrados debería existir alguna relación que de lugar a es valor constante. Para buscar esta relación nos centraremos únicamente en partículas individuales en un rango de velocidades iniciales \( v_0 \in [0.001, 0.9] \) con un mismo valor para la pared (Figura \ref{fig:ajuste_parametros_1D_relativity}). 

\begin{figure}[H]
    \begin{subfigure}[b]{0.5\textwidth}
        \centering
        \includegraphics[scale=0.55]{images/Simulaciones/Relatividad_1D/ajuste_parametros_W-001.pdf}
        \caption{$w_r = -0.01 \quad w_l = 0$}
    \end{subfigure}
    \hfill
    \begin{subfigure}[b]{0.5\textwidth}
        \centering
        \includegraphics[scale=0.55]{images/Simulaciones/Relatividad_1D/ajuste_parametros_W-01.pdf}
        \caption{$w_r = -0.1 \quad w_l = 0$}
    \end{subfigure}
    \caption{Ajuste de los parámetros que ajustan los datos simulados para una misma velocidad de pared con respecto a un rango de velocidades la partícula. La función de ajuste utilizada es una función de segundo grado: \( ax^2 + bx + c \).}
    \label{fig:ajuste_parametros_1D_relativity}
\end{figure}

Cuando la velocidad de las paredes es constante podemos ver un comportamiento claro en los parámetros en función de la velocidad, pero cada velocidad de pared va a generar distintos parámetros y no se podría obtener una relación clara.

\vspace{3mm}

\begin{figure}[H]
    \begin{subfigure}[b]{0.5\textwidth}
        \centering
        \includegraphics[scale=0.55]{images/Simulaciones/Relatividad_1D/ajuste_parametros_V-001.pdf}
        \caption{$v_0 = 0.01$}
    \end{subfigure}
    \hfill
    \begin{subfigure}[b]{0.5\textwidth}
        \centering
        \includegraphics[scale=0.55]{images/Simulaciones/Relatividad_1D/ajuste_parametros_V-02.pdf}
        \caption{$v_0 = 0.2$}
    \end{subfigure}
    \caption{Ajuste de los parámetros que ajustan los datos simulados para una misma velocidad de pared con respecto a un rango de velocidades la partícula. La función de ajuste utilizada es una función de segundo grado: \( ax^2 + bx + c \).}
    \label{fig:ajuste_parametros_1D_relativity_Vcte}
\end{figure}

Realizando el mismo procedimiento pero con las paredes, es decir, se simulan partículas con una misma velocidad en billares con velocidad en los muros \( w \in [0.001, 0.9] \) (Figura \ref{fig:ajuste_parametros_1D_relativity_Vcte}). Se observa como la función que ajusta los parámetros es muy diferente a la empleada anteriormente y se tenga que buscar una función con caracterísitcas más especiales. Además si se comparan los cuatro casos mostrados, podemos ver que en el caso de velocidad de la pared constante la forma de los parámetros son muy similares (aunque con valores distintos), en cambio para la velocidad del muro constante el único parámetro que parece mantener la forma es \( \beta \). El resto de parámetros se van aproximando a una función constante a medida que el valor de la velocidad de la partícula se ve aumentado.

\vspace{3mm}

Procediendo del mismo modo pero ahora con las colisiones inelásticas (Figura \ref{fig:ajuste_1D_inelastic}) si existe algúna relación que pueda aproximar de manera general el billar. 

\begin{figure}[H]
    \begin{subfigure}[b]{0.5\textwidth}
        \centering
        \includegraphics[scale=0.55]{images/Simulaciones/Inelastico_1D/1D_A-N1000_fit.pdf}
        \caption{$w_r = -0.001 \quad w_l = 0 \quad \epsilon = 0.99$}
    \end{subfigure}
    \hfill
    \begin{subfigure}[b]{0.5\textwidth}
        \centering
        \includegraphics[scale=0.55]{images/Simulaciones/Inelastico_1D/1D_B-N1000_fit.pdf}
        \caption{$w_r = -0.01 \quad w_l = 0 \quad \epsilon = 0.99$}
    \end{subfigure}
    \hfill
    \begin{subfigure}[b]{0.5\textwidth}
        \centering
        \includegraphics[scale=0.55]{images/Simulaciones/Inelastico_1D/1D_C-N1000_fit.pdf}
        \caption{$w_r = -0.01 \quad w_l = 0.01 \quad \epsilon = 0.7$}
    \end{subfigure}
    \hfill
    \begin{subfigure}[b]{0.5\textwidth}
        \centering
        \includegraphics[scale=0.55]{images/Simulaciones/Inelastico_1D/1D_D-N1000_fit.pdf}
        \caption{$w_r = -0.3 \quad w_l = 0.3 \quad \epsilon = 0.99$}
    \end{subfigure}
    \caption{Ajuste de los datos a la ecuación \ref{eq:acumulativa_general}. Los valores \( w_l \) y \( w_r \) corresponden a la velocidad de la pared derecha e izquierda, respectivamente. Todas las colisiones son  inelásticas.}
    \label{fig:ajuste_1D_inelastic}
\end{figure}

\vspace{3mm}

Comprobamos que ninguno de los parámetros se ha mantenido igual a los elásticos, ni tampoco el valor que se veía constante \( \gamma \) lo es ahora. Con esta simple visualización se puede afirmar sin realizar un mayor estudio que para un billar inelástico tampoco podemos encontrar ninguna función general que nos diga las velocidad en la n-esima colisión. 


\section{Dos dimensiones}

Pasemos ahora a un espacio de dos dimensiones donde la partícula puede moverse en ambos ejes y el billar está compuesto por cuatro paredes. A priori los resultados que se esperan obtener son similares a las simulaciones realizadas para una dimensión y la teoría desarrollada, pero esta vez nos interesa si el área disminuye o aumenta.

\vspace{3mm}

Las paredes del billar varían su posición según una función lineal, ya que, su velocidad es siempre contante. Eso implica que su área se ve modificada de manera cuadrática y pueden darse varias combinaciones: aumentar-disminuir, disminuir-aumentar, aumentar, disminuir. Es por ello que se podría esperar una variación de la velocidad media de las partículas según la función del área que nos encontremos. 

\subsection{Clasico}

\begin{figure}[H]
    \begin{subfigure}[b]{0.5\textwidth}
        \centering
        \includegraphics[scale=0.55]{images/Simulaciones/Clasico_2D/2D_A-N1000.pdf}
        \caption{Velocidad media de las partículas.}
    \end{subfigure}
    \hfill
    \begin{subfigure}[b]{0.5\textwidth}
        \centering
        \includegraphics[scale=0.55]{images/Simulaciones/Clasico_2D/2D_A-N1000_Area.pdf}
        \caption{Área del billar.}
    \end{subfigure}
    \caption{Resultados obtenidos para un billar con área siempre en aumento.}
    \label{fig:2D_A-N1000_con_Area}
\end{figure}

\begin{figure}[H]
    \begin{subfigure}[b]{0.5\textwidth}
        \centering
        \includegraphics[scale=0.55]{images/Simulaciones/Clasico_2D/2D_B-N1000.pdf}
        \caption{Velocidad media de las partículas.}
    \end{subfigure}
    \hfill
    \begin{subfigure}[b]{0.5\textwidth}
        \centering
        \includegraphics[scale=0.55]{images/Simulaciones/Clasico_2D/2D_B-N1000_Area.pdf}
        \caption{Área del billar.}
    \end{subfigure}
    \caption{Resultados obtenidos para un billar con área siempre en disminución.}
    \label{fig:2D_B-N1000_con_Area}
\end{figure}

En los casos donde el área sigue una función lineal (Figura \ref{fig:2D_A-N1000_con_Area} y \ref{fig:2D_B-N1000_con_Area}) obtenemos el mismo resultado que en caso de una dimensión. La velocidad tiende a aumentar hasta el infinito sin establecer un valor constante ni finito cuando el área se ve amentada, y al verse disminuida la velocidad media se reduce hasta un valor estable donde no colisionará más con ninguna de las paredes que se mueven (o con ninguna otra si su movimiento forma un ángulo recto con alguna de esas paredes móviles).

\vspace{3mm}

En los casos donde todas las paredes se mueven, y lo hacen con velocidades distintas, (Figura \textcolor{red}{Añadir figura}) observamos como la velocidad media aumenta siempre a pesar de que el área aumenta y disminuye. Cabría pensar que lo lógico según lo visto hasta ahora es una disminución de la velocidad y luego un aumento en ella siguiendo la inversa del área. El motivo por el que no ocurre es por como es calculada el área, es decir, si hacemos un desarrollo matemático llegamos a 

\begin{equation}
    A(t) = \left[ a_0 + t (w_R - w_L) \right] \cdot \left[ b_0 + t (w_T - w_B) \right]
\end{equation}

donde \( w_i \) son las velocidades de las paredes (Top, Bottom, Left, Right en inglés), \( a_0 \) es la distancia inicial entre las paredes horizontales, y \( b_0 \) la distancia inicial entre las verticales. Vemos entonces que es posible definir una velocidad de las paredes equivalente \( w_H =  w_R - w_L \) y \( w_V = w_T - w_B \), con lo que se conseguiría un billar donde sólo se desplazaran dos paredes. Dentro de este billar equivalente, si al menos una de ellas se mueve hacia el interior del billar, en algún momento esa pared colisionará con la partícula lo que provocaría el aumento de su velocidad. Para verlo mejor imaginemos que una partícula con la única componente de velocidad no nula es la vertical donde la pared superior (Top) se aleja del centro, y la pared izquierda se dirige hacia el interior. Si la pared no se moviera, la partícula perdería velocidad hasta no colisionar con ninguna de las paredes, pero al no ser el caso, en algún momento la partícula obtendrá una velocidad no nula en la componente horizontal originando varias colisiones con las paredes laterales. Además, la partícula solamente colisionará con los muros laterales dado que seguirá perdiendo velocidad en la componente vertical pero no así su velocidad total, la cual se verá aumentada tras cada colisión con el muro izquierdo.

\vspace{3mm}

Con este ejemplo se afirma que en el caso de tener al menos una pared con movimiento hacia el interior del billar y esta posea una velocidad mayor a su paralela, independientemente del aumento o disminución del área, la partícula verá aumentada su velocidad total y se verá una acelaración de Fermi. Es importante notar que la pared que se dirige al interior debe tener una velocidad mayor a su pared paralela por el posible caso mostrado en la figura \ref{fig:clasico_distancia_infinito}, donde dos paredes paralelas se dirigian al centro y otra hacia el exterior ocasionando una disminución de la velocidad de la partícula.

\textcolor{red}{Falta poner foto clasicas del area con forma cuadratica. Se deberían simular hasta 10.000 colisiones. Donde pone AÁDIR FIGURA se refiere a las graficas que se hablan aqui.}

\textcolor{blue}{Cuando se hayan puesto las graficas dichas en rojo, seguir con relativista y ver si sigue la misma línea}

\subsection{Relativista}

\subsection{Colisión inelástica}

\subsection{Análisis de los resultados}

\end{document}