\documentclass[11pt, spanish]{book}
\usepackage{MiEstilo}

\begin{document}

\chapter{Resultados numéricos}

Para diversos casos donde las paredes se mueven mover o no, se va realizar la simulación de un partícula en el interior de un billar rectangular. Para ello el procedimiento es usar el código creado \textcolor{red}{Referenciar código mio} para distintos grupos de \( 10^3 \) partículas, no interactuantes entre sí, siguiendo las siguientes pautas:

\begin{enumerate}
    \item Cada grupo tiene el mismo módulo de velocidad pero con posiciones y direcciones de velocidades aleatorias para cada partícula.
    \item Las condiciones iniciales de cada pared son las mismas para todos los grupos
    \item El resultado final corresponde a la velocidad media de todas las partículas tras cada colisión con alguna de las paredes.
    \item En caso de no obtener más colisiones con nigún muro, se mantendrá la última velocidad obtenida para el cálculo de la media en cada colisión del resto de partículas.
\end{enumerate}

\section{Caso unidimensional}

Vamos a analizar la situación donde la partícula se encuentra entre dos paredes moviéndose en un sólo eje.

\subsection{Clasico}
\begin{enumerate}
    \item Area expandir 
    \item Area contraer
    \item  Paredes oscilatorias
    \item Colisión inelástica
\end{enumerate}
\subsection{Relativista}
\begin{enumerate}
    \item Area expandir
    \item Area contraer
    \item  Paredes oscilatorias
    \item Colisión inelástica
\end{enumerate}
\section{Dos dimensiones}

\subsection{Clasico}
\begin{enumerate}
    \item Area expandir
    \item Area contraer
    \item Paredes oscilatorias
    \item Colisión inelástica
\end{enumerate}
\subsection{Relativista}
\begin{enumerate}
    \item Area expandir
    \item Area contraer
    \item Paredes oscilatorias
    \item Colisión inelástica
\end{enumerate}


\end{document}