\documentclass[11pt, spanish]{book}
\usepackage{MiEstilo}

\begin{document}

\chapter{Resultados numéricos}

Para diversos casos donde las paredes se mueven mover o no, se va realizar la simulación de un partícula en el interior de un billar rectangular. Para ello el procedimiento es usar el código creado \textcolor{red}{Referenciar código mio} para \( 10^3 \) partículas, no interactuantes entre sí. De este modo es similar a realizar una simulación de un gas ideal donde cada partícula inicia con una velocidad y posición distinta al resto dentro del mismo recinto. Con los datos obtenidos se puede obtener la velocidad media tras cada colisión y comprobar la existencia de la aceleración de Fermi.

\begin{figure}[H]
    \begin{subfigure}[b]{0.5\textwidth}
        \centering
        \includegraphics[scale=0.35]{images/Billiard_general.pdf}
        \caption{$w_i<0$}
        \label{fig:a}
    \end{subfigure}
    \hfill
    \begin{subfigure}[b]{0.5\textwidth}
        \centering
        \includegraphics[scale=0.35]{images/Billiards_generico_2.pdf}
        \caption{$w=0$}
        \label{fig:s}
    \end{subfigure}
    \caption{Trayectoria de una partícula. a) En un billar con paredes dirigiéndose hacia el centro. b) Paredes con velocidad nula}
\end{figure}


Los billares que interesan simular son aquellos que tienen al menos una de las paredes con velocidad no nula, ya que, es así cuando la velocidad de la partícula se puede ver modificada. Además para realizar una mejor comparación vamos a separar los casos donde la partícula se encuentra en un espacio de una dimensión y el de dos dimensiones, comparando ambas teorías en sus correspondiente dimensión, y por otro lado que ocurre cuando las colisiones no son totalmente elásticas.

\section{Caso unidimensional}

Nos situamos en un espacio donde podemos suponer que las paredes son dos puntos móviles donde la partícula puede desplazarse sobre la línea que los une. Dado que el cambio del módulo de la velocidad en la partícula es debido al movimiento de las paredes, la distancia \( d(t) \) va a dejar de ser constante. De otro modo, la partícula oscilaría entre dos paredes moviéndose en la misma dirección y misma velocidad donde la partícula aumentaría y disminuiría su velocidad en un mismo factor. 

\subsection{Clasico}

Según la teoría desarrollada en el capítulo anterior cuando al menos una de las paredes se mueve de forma que su movimiento tiende a aumentar el área del billar, en este caso la distancia entre las paredes se ve aumentada, entonces la velocidad disminuye. Esto se ve también en las simulaciones (Figura \ref{fig:clasico_1D_A}) llegando además a una velocidad donde la partícula ve modificada su velocidad hasta una menor que la velocidad de la pared a la que se dirige.

\begin{figure}[H]
    \centering
    \includegraphics[scale=0.65]{images/Simulaciones/Clasico_1D/1D_A-N1000.pdf}
    \caption{Velocidades medias tras cada colisión en una dimensión. La distancia entre paredes aumenta}
    \label{fig:clasico_1D_A}
\end{figure}


\begin{figure}[H]
    \centering
    \includegraphics[scale=0.65]{images/Simulaciones/Clasico_1D/1D_B-N1000.pdf}
    \caption{Velocidades medias tras cada colisión en una dimensión. La distancia entre paredes disminuye}
    \label{fig:clasico_1D_B}
\end{figure}

Si nos situamos en el caso contrario, donde la distancia entre paredes se ve disminuida, vemos que los resultados son inversos (Figura \ref{fig:clasico_1D_B}). La velocidad tiende hacia el infinito sin aproximarse a un ningún valor al igual que se podía ver en las Cobwebs (Figura \ref{fig:Cobweb_Classic_1D} \textcolor{red}{En otro capitulo}). 

\vspace{3mm}

Es interesante ver también los casos donde las paredes tienen velocidades distintas donde una intenta ampliar la distancia y la otra acortarla. 

\begin{figure}[H]
    \begin{subfigure}[b]{0.5\textwidth}
        \centering
        \includegraphics[scale=0.55]{images/Simulaciones/Clasico_1D/1D_C1-N100.pdf}
        \caption{$d \rightarrow \infty$}
        \label{fig:clasico_distancia_infinito}
    \end{subfigure}
    \hfill
    \begin{subfigure}[b]{0.5\textwidth}
        \centering
        \includegraphics[scale=0.55]{images/Simulaciones/Clasico_1D/1D_C2-N1000.pdf}
        \caption{$d \rightarrow 0$}
        \label{fig:clasico_distancia_cero}
    \end{subfigure}
    \caption{Velocidades medias según la distancia entre paredes.}
\end{figure}

Se observa como el ampliar o reducir la distancia cambia totalmente el sistema. En el primer caso (Figura \ref{fig:clasico_distancia_cero}) las distancia entre las paredes aumenta lentamente y se ve como es necesario varias colisiones antes de conseguir una velocidad estable. Por el otro lado (Figura \ref{fig:clasico_distancia_infinito}), la distancia se reduce y su velocidad media aumenta tras cada colisión sin llegar a un valor estable, además de tender hacia el infinito. 

\vspace{3mm}

Con estas últimas simulaciones se puede afirmar que dado una billar con ambas paredes en movimiento donde las paredes se persiguen es posible encontrar otro billar donde únicamente una donde las paredes sea móvil y obtener los mismos resultados. En el caso que las paredes posean direcciones distintas también será posible encontrar un billar equivalente dado que la velocidad cambia tras cada colisión y sólo es necesario encontrar una velocidad para la pared que tras cada colisión aumente o disminuya la velocidad un factor que queramos. 

\subsection{Relativista}

Si nos situamos ahora en el contexto relativista sabemos que no es posible superar la velocidad para ninguna partícula, además de haberlo comprobado teóricamente en el capítulo anterior. Realizando la mismas simulaciónes pero con velocidades en función de la velocidad de la luz \( u \in [0, 0.5c] \), vemos como los resultados tienen una forma similar a la teoría clásica. Cuando se tiene un muro en reposo y otro con un movimiento con tendencia a aumentar la distancia, la velocidad media disminuye hasta un valor constante (Figura \ref{fig:relatividad_1D_A}). 

\vspace{3mm}

El caso opuesto donde las paredes se distancian (Figura \ref{fig:relatividad_1D_B}), se observa también un aumento de la velocidad pero aquí si vemos el límite que toma, el de la velocidad de la luz. Cuando la pared realiza un movimiento ``lento'' no se consigue alcanzar una velocidad casi a la de la luz hasta 3.000 colisiones después, en cambio con una velocidad mayor (10\% el de la luz) apenas se necesitan 20 colisiones. Vemos así la importancia que obtiene la velocidad de las paredes para obtener una velocidad elevada en poco tiempo.

\vspace{3mm}

De nuevo podemos ver diferentes escenarios con ambas paredes móviles. 

\begin{figure}[H]
    \centering
    \includegraphics[scale=0.65]{images/Simulaciones/Relatividad_1D/1D_A-N1000.pdf}
    \caption{Velocidades medias tras cada colisión en una dimensión. La distancia entre paredes aumenta}
    \label{fig:relatividad_1D_A}
\end{figure}

\begin{figure}[H]
    \centering
    \includegraphics[scale=0.65]{images/Simulaciones/Relatividad_1D/1D_B-Juntos-N1000.pdf}
    \caption{Velocidades medias tras cada colisión en una dimensión. La distancia entre paredes aumenta}
    \label{fig:relatividad_1D_B}
\end{figure}



\subsection{Colisión inelástica}

\section{Dos dimensiones}

\subsection{Clasico}

\subsection{Relativista}

\subsection{Colisión inelástica}

\end{document}