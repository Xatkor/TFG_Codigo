\documentclass[11pt, spanish]{book}
\usepackage{MiEstilo}

\begin{document}

\chapter{Resultados numéricos}

Para diversos casos donde las paredes se mueven mover o no, se va realizar la simulación de un partícula en el interior de un billar rectangular. Para ello el procedimiento es usar el código creado \textcolor{red}{Referenciar código mio} para \( 10^3 \) partículas, no interactuantes entre sí. De este modo es similar a realizar una simulación de un gas ideal donde cada partícula inicia con una velocidad y posición distinta al resto dentro del mismo recinto. Con los datos obtenidos se puede obtener la velocidad media tras cada colisión y comprobar la existencia de la aceleración de Fermi.

\begin{figure}[H]
    \begin{subfigure}[b]{0.5\textwidth}
        \centering
        \includegraphics[scale=0.35]{images/Billiard_general.pdf}
        \caption{$w_i<0$}
        \label{fig:a}
    \end{subfigure}
    \hfill
    \begin{subfigure}[b]{0.5\textwidth}
        \centering
        \includegraphics[scale=0.35]{images/Billiards_generico_2.pdf}
        \caption{$w=0$}
        \label{fig:s}
    \end{subfigure}
    \caption{Trayectoria de una partícula. a) En un billar con paredes dirigiéndose hacia el centro. b) Paredes con velocidad nula}
\end{figure}


Los billares que interesan simular son aquellos que tienen al menos una de las paredes con velocidad no nula, ya que, es así cuando la velocidad de la partícula se puede ver modificada. Además para realizar una mejor comparación vamos a separar los casos donde la partícula se encuentra en un espacio de una dimensión y el de dos dimensiones, comparando ambas teorías en sus correspondiente dimensión, y por otro lado que ocurre cuando las colisiones no son totalmente elásticas.

\section{Caso unidimensional}

Nos situamos en un espacio donde las partículas están confinadas entre las paredes de un billar rectangular, y donde tiene su movimiento está restringido a la línea recta que une dos paredes paralelas. Con esta restricción y la posibilidad de que esas paredes paralelas puedan moverse sobre esa misma línea vamos a ver si los resultados numéricos afirman la teoría.

\vspace{3mm}

Con estas condiciones vamos a simular situaciones donde la distancia \( d(t) \) va a dejar de ser constante. De otro modo, la partícula oscilaría entre dos paredes moviéndose en la misma dirección y misma velocidad donde la partícula aumentaría y disminuiría su velocidad en un mismo factor y no habría interés en su estudio.

\subsection{Clasico}

Según la teoría desarrollada en el capítulo anterior cuando al menos una de las paredes se mueve de forma que su movimiento tiende a aumentar el área del billar, en este caso la distancia entre las paredes se ve aumentada, entonces la velocidad disminuye. Esto se ve también en las simulaciones (Figura \ref{fig:clasico_1D_A}) llegando además a una velocidad donde la partícula ve modificada su velocidad hasta una menor que la velocidad de la pared a la que se dirige.

\begin{figure}[H]
    \centering
    \includegraphics[scale=0.65]{images/Simulaciones/Clasico_1D/1D_A-N1000.pdf}
    \caption{Velocidades medias tras cada colisión en una dimensión. La distancia entre paredes aumenta}
    \label{fig:clasico_1D_A}
\end{figure}


\begin{figure}[H]
    \centering
    \includegraphics[scale=0.65]{images/Simulaciones/Clasico_1D/1D_B-N1000.pdf}
    \caption{Velocidades medias tras cada colisión en una dimensión. La distancia entre paredes disminuye}
    \label{fig:clasico_1D_B}
\end{figure}

Si nos situamos en el caso contrario, donde la distancia entre paredes se ve disminuida, vemos que los resultados son inversos (Figura \ref{fig:clasico_1D_B}). La velocidad tiende hacia el infinito sin aproximarse a un ningún valor al igual que se podía ver en las Cobwebs (Figura \ref{fig:Cobweb_Classic_1D} \textcolor{red}{En otro capitulo}). 

\vspace{3mm}

Es interesante ver también los casos donde las paredes tienen velocidades distintas donde una intenta ampliar la distancia y la otra acortarla. 

\begin{figure}[H]
    \begin{subfigure}[b]{0.5\textwidth}
        \centering
        \includegraphics[scale=0.55]{images/Simulaciones/Clasico_1D/1D_C1-N100.pdf}
        \caption{$d \rightarrow \infty$}
        \label{fig:clasico_distancia_infinito}
    \end{subfigure}
    \hfill
    \begin{subfigure}[b]{0.5\textwidth}
        \centering
        \includegraphics[scale=0.55]{images/Simulaciones/Clasico_1D/1D_C2-N1000.pdf}
        \caption{$d \rightarrow 0$}
        \label{fig:clasico_distancia_cero}
    \end{subfigure}
    \caption{Velocidades medias según la distancia entre paredes.}
\end{figure}

Se observa como el ampliar o reducir la distancia cambia totalmente el sistema. En el primer caso (Figura \ref{fig:clasico_distancia_cero}) las distancia entre las paredes aumenta lentamente y se ve como es necesario varias colisiones antes de conseguir una velocidad estable. Por el otro lado (Figura \ref{fig:clasico_distancia_infinito}), la distancia se reduce y su velocidad media aumenta tras cada colisión sin llegar a un valor estable, además de tender hacia el infinito. 

\vspace{3mm}

Con estas últimas simulaciones se puede afirmar que dado una billar con ambas paredes en movimiento donde las paredes se persiguen es posible encontrar otro billar donde únicamente una donde las paredes sea móvil y obtener los mismos resultados. En el caso que las paredes posean direcciones distintas también será posible encontrar un billar equivalente dado que la velocidad cambia tras cada colisión y sólo es necesario encontrar una velocidad para la pared que tras cada colisión aumente o disminuya la velocidad un factor que queramos. 

\subsection{Relativista}

Si nos situamos ahora en el contexto relativista sabemos que no es posible superar la velocidad para ninguna partícula, además de haberlo comprobado teóricamente en el capítulo anterior. Realizando la mismas simulaciónes pero con velocidades en función de la velocidad de la luz \( u \in [0, 0.5c] \), vemos como los resultados tienen una forma similar a la teoría clásica. Cuando se tiene un muro en reposo y otro con un movimiento con tendencia a aumentar la distancia, la velocidad media disminuye hasta un valor constante (Figura \ref{fig:relatividad_1D_A}). 

\vspace{3mm}

El caso opuesto donde las paredes se distancian (Figura \ref{fig:relatividad_1D_B}), se observa también un aumento de la velocidad pero aquí si vemos el límite que toma, el de la velocidad de la luz. Cuando la pared realiza un movimiento ``lento'' no se consigue alcanzar una velocidad casi a la de la luz hasta 3.000 colisiones después, en cambio con una velocidad mayor (10\% el de la luz) apenas se necesitan 20 colisiones. Vemos así la importancia que obtiene la velocidad de las paredes para obtener una velocidad elevada en poco tiempo.

\vspace{3mm}

\begin{figure}[H]
    \centering
    \includegraphics[scale=0.65]{images/Simulaciones/Relatividad_1D/1D_A-N1000.pdf}
    \caption{Velocidades medias tras cada colisión en una dimensión. La distancia entre paredes aumenta}
    \label{fig:relatividad_1D_A}
\end{figure}

\begin{figure}[H]
    \centering
    \includegraphics[scale=0.65]{images/Simulaciones/Relatividad_1D/1D_B-Juntos-N1000.pdf}
    \caption{Velocidades medias tras cada colisión en una dimensión. La distancia entre paredes aumenta}
    \label{fig:relatividad_1D_B}
\end{figure}

Se pueden observar distintos escenarios con ambas paredes móviles. En primer lugar (Figura \ref{fig:relatividad_1D_C}), la distancia entre las paredes aumenta, lo que resulta en una velocidad media descendente hasta que se alcanza una velocidad constante. Posteriormente, cuando la distancia entre las paredes disminuye (Figura \ref{fig:relatividad_1D_C2}), se observa nuevamente un aumento en la velocidad, aproximándose a la velocidad de la luz.

\vspace{3mm}

Se puede notar la diferencia que existe entre la rapidez a la que la velocidad crece y decrece. Cuando la distancia aumenta la velocidad pérdida en 500 colisiones es de un \( 60\% \) mientras que al disminuir la distancia en el mismo número de colisiones aumenta en un \( 586\% \). Esto hace ver la gran aceleración que se ve producida aún teniendo la tasa de cambio de la distancia en un mismo factor (las paredes interecambian velocidades de un caso a otro).

\begin{figure}[H]
    \centering
    \includegraphics[scale=0.65]{images/Simulaciones/Relatividad_1D/1D_C-N1000.pdf}
    \caption{Velocidades medias tras cada colisión en una dimensión. La distancia entre paredes aumenta}
    \label{fig:relatividad_1D_C}
\end{figure}

\begin{figure}[H]
    \centering
    \includegraphics[scale=0.65]{images/Simulaciones/Relatividad_1D/1D_C2-N1000.pdf}
    \caption{Velocidades medias tras cada colisión en una dimensión. La distancia entre paredes aumenta}
    \label{fig:relatividad_1D_C2}
\end{figure}

\subsection{Colisión inelástica}

Hasta ahora hemo visto como tanto para el caso clásico como para el relativista obtenemos una aceleración cuando la distancia entre paredes se ve aumentada. Por lo tanto estamos en presencia de una aceleración de Fermi. En el mundo real esta aceleración no podría darse dado que entre las sucesivas colisiones las partículas se ven sometidas a distintas fuerzas que reducen la velociadad (p.e. fuerza de fricción). 

\vspace{3mm}

De las distintas formas que pueden suprimir la aceleración vamos a utlizar la pérdida de energía que se produce en cada colisión, así estaríamos ante colisiones inelásticas. Recordemos que esta supresión sólo se aplica al caso relativista, ya que, es el que más interés causa al ser el más cercano a la realidad.

\vspace{3mm}

Se debe tener en cuenta que las soluciones obtenidas en el capítulo anterior suponen colisiones sucesivas con una pared con la misma velocidad y siempre en sentido contrario a la dirección de la partícula. Hasta ahora sólo se han visto billares donde una pared era móvil o tenían una velocidad cercana entre sí. Esto era así por la equivalencia que se podía encontrar entre billares para llegar a una velocidad muy cercana a la de la luz sin importar el número de colisiones que eran necesarias para llegar a ese valor. Estos casos estudiados teóricamente se pueden ver en las figuras \ref{fig:1D_inelastic_A} y \ref{fig:1D_inelastic_B}. Podemos notar como una colisión casi inelástica reduce drásticamente la velocidad máxima posible de la partícula para una misma velocidad dada.

\begin{figure}[H]
    \begin{subfigure}[b]{0.5\textwidth}
        \centering
        \includegraphics[scale=0.55]{images/Simulaciones/Inelastico_1D/1D_A-N1000.pdf}
        \caption{$\epsilon = 0.99$}
    \end{subfigure}
    \hfill
    \begin{subfigure}[b]{0.5\textwidth}
        \centering
        \includegraphics[scale=0.55]{images/Simulaciones/Inelastico_1D/1D_C-N1000.pdf}
        \caption{$\epsilon = 0.7$}
    \end{subfigure}
    \caption{Velocidades medias según la distancia entre paredes.}
    \label{fig:1D_inelastic_A}
\end{figure}

\begin{figure}[H]
    \begin{subfigure}[b]{0.5\textwidth}
        \centering
        \includegraphics[scale=0.55]{images/Simulaciones/Inelastico_1D/1D_B-N1000.pdf}
        \caption{$\epsilon = 0.99$}
    \end{subfigure}
    \hfill
    \begin{subfigure}[b]{0.5\textwidth}
        \centering
        \includegraphics[scale=0.55]{images/Simulaciones/Inelastico_1D/1D_D-N1000.pdf}
        \caption{$\epsilon = 0.7$}
    \end{subfigure}
    \caption{Velocidad media de las partículas tras colisiones inelásticas con paredes que se mueven hacia el interior del billar a la misma velocidad.}
    \label{fig:1D_inelastic_B}
\end{figure}

\subsection{Ajuste de los resultados}

Como se ha podido ver las gráficas mostradas se asemejan a funciones contínuas. En los casos donde aparecía la aceleración de Fermi se pueden aproximar a funciones sigmoide

\begin{equation}
    f(n) = \dfrac{\beta}{\beta + e^{-n\alpha}}
\end{equation}

e incluso con mayor exactitud a una función de distribuión acumulativa

\begin{equation}
    f(n) = 1 - \gamma\exp(-\beta n^\alpha)
\end{equation}

donde \( \alpha, \beta, \gamma \) son constantes a determinar para cada billar. 

\textcolor{purple}{La función que mejor parece ajustarse a simple vista es \( f(x)=1 - a\cdot\exp(c\cdot x^b) \). Es una distribución acumulativa. Los valores de \( (a, b, c) \) deberína ser a = velocidad inicial, c = velocidad de la pared, b = una constante, x = numero de colisiones.} 

\textcolor{blue}{Ajustar las graficas a las funciones expuestas}


Los casos donde no se puede hacer un desarrollo matemático se muestran en las figuras \textcolor{blue}{Poner las figuras de la E-\ldots y comentarlas}.

\textcolor{blue}{No llega a una velocidad máxima fija sino que oscila entre dos valores. Cuando la línea es recta significa que el billar ha terminado con el area A=0 y no simula más}

\section{Dos dimensiones}

\subsection{Clasico}

\subsection{Relativista}

\subsection{Colisión inelástica}

\end{document}