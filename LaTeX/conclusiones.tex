\documentclass[11pt, spanish]{book}
\usepackage{MiEstilo}

\begin{document}

\chapter{Conclusiones}

Usando los conocimientos aprendidos en el grado sobre la dinámica clásica y relativista se ha podido obtener la ley de movimiento de una partícula según el marco teórico en el que nos situemos. Con ello se ha podido observar como aparece un tipo de aceleración que no parece tener límite dentro de la física clásica, pero que se ha podido ver que en la teoría relativista dichi límite sí existe. Con mayor conocimiento matemático y tiempo sería interesante analizar en mayor profundidad el espacio en dos dimensiones y ver si también se puede obtener una expresión cerrada para la velocidad límite. 

\vspace{3mm}

Para aquellos casos donde se aplicó la colisión inelástica para frenar la aceleración de Fermi se ha observado de manera sorprendente la importancia que tiene el coeficiente de restitución. Una ligera modificación de este coeficiente limita en gran medida la velocidad que puede alcanzar la partícula contenida en el billar. Además para el caso relativista se ha implementado el uso de una función acumulativa que ha conseguido ajutar casi a la perfección la velocidades en cada colisión en todos los casos donde aparecía dicha aceleración.

\vspace{3mm}

Finalmente, en las primeras pruebas del código utilizado para la simulación se observaba como aparecía la aceleración de Fermi en aquellos billares donde el área tendía a hacerse nula. Con esa simple observación se postuló que el área tenía bastante implicación en la existencia de la aceleración, y con el estudio mostrado en el trabajo se puede afirmar que en billares rectangulares, en una o dos dimensiones, la disminución del área o distancia entre paredes va agenerar que la velocidad de la partícula alcance velocidades muy altas.  

\end{document}