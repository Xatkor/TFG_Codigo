%\documentclass[11pt, spanish]{book}
%\usepackage{MiEstilo}
%
%\begin{document}

\chapter{Conclusiones}

Se ha podido observar la gran diferencia que existe entre la teoría clásica y la relativista. La primera obtiene resultados que no convergen en aquellos casos donde la aceleración de Fermi aparece, mientras que la segunda converge a un valor máximo de velocidad igual al de la luz cuando las colisiones son elásticas. 

\vspace{3mm}

Dentro de los resultados relativistas se ha visto como obtener una expresión para la velocidad máxima que puede adquirir la partícula en el caso de estar en el espacio de una dimensión, que además obtiene resulatdos idénticos a las simulaciones realizadas. En cambio, en el espacio de dos dimensiones su obtención requiere un mayor desarrollo matemático para encontrar esa expresión (siempre y cuando sea posible su expresión analítica). A pesar de esta diferencia entre ambas, se afirma para ambas dimensiones que la aparición de la aceleración de Fermi es consecuencia (para los billares cuadrilateros) de la disminución de la distancia o del área, según sea el espacio donde nos encontremos. 

%\end{document}