%\documentclass[11pt, spanish]{book}
%\usepackage{MiEstilo}
%
%
%\begin{document}
%
\chapter{Dinámica de la partícula en el billar}

\section{Billar clásico}

Situándonos en la teoría clásica vamos a suponer un billar cuyas paredes tienen masa infinita y donde las colisiones producidas son totalmente elásticas. Procedamos a ver como se modifica la velocidad de la partícula en el momento de la colisión.

\vspace{3mm}

Al ser una colisión elástica el ángulo de reflexión será igual al ángulo con el que llega la partícula al punto de colisión. En el caso de tener paredes estáticas y una partícula con momento inicial \( \mathbf{p} = m\mathbf{u}\), el cambio que se produce en el momento es 

\begin{equation}\label{eq:elastic_col}
    \mathbf{p}' = \mathbf{p} - 2\mathbf{n}(\mathbf{p} \cdot \mathbf{n})
\end{equation}

Podemos notar que el cambio del momento se produce únicamente en la dirección normal a la pared dado el producto \( (\mathbf{p} \cdot \mathbf{n}) \), que es nulo cuando son vectores perpendiculares y deja el momento invariante.

\begin{figure}[H]
    \centering
    \includegraphics[scale=1.5]{images/elastic_collision.pdf}
    \caption{Cambio de momento tras una colisión elástica}
    \label{fig:elastic_col}
\end{figure}

\vspace{3mm}

Supongamos ahora que tenemos una partícula con momento \( \mathbf{p} \) y una pared también en movimiento con momento \( \mathbf{P} \) con dirección a la normal. Para obtener el cambio de momento tras la colisión nos trasladamos al sistema de referencia donde la pared está en reposo, así podemos aplicar la ecuación \ref{eq:elastic_col} y volver al sistema de referencia inicial.

\begin{figure}[H]
    \centering
    \includegraphics[scale=1.5]{images/cambio_sistema_referencia.pdf}
    \caption{Posición de la pared y la partícula en los sistemas $S_0$ y $S$}
    \label{fig:cambio_referencia}
\end{figure}

\vspace{3mm}

El sistema de referencia S sitúa a la pared en su origen y se mueve con velocidad \( \mathbf{w} \) (igual a la del muro). La velocidad relativa de la partícula en ese nuevo sistema es

\begin{equation}
    \mathbf{v}_S = \mathbf{u} - \mathbf{w}
\end{equation}

La nueva velocidad tras colisionar será

\begin{equation}
    \mathbf{v}_S' = \mathbf{v}_S - 2\mathbf{n}(\mathbf{v}_S \cdot \mathbf{n})
\end{equation}

Ahora deshaciendo la transformación de la velocidad y volviendo al sistema \( S_0 \), se obtiene

\begin{align}
    \mathbf{u}' &= \mathbf{v}_S' + \mathbf{w} \\
    &= \mathbf{v}_S - 2\mathbf{n}(\mathbf{v}_S \cdot \mathbf{n}) + \mathbf{w} \\
    &= \mathbf{u} - 2\mathbf{n}\left[(\mathbf{u} - \mathbf{w}) \cdot \mathbf{n}\right]
\end{align}

\vspace{3mm}

Se puede ver que en el caso más simple de una colisión frontal contra una pared en reposo la velocidad que adquiere la partícula es igual a 

\begin{equation}\label{eq:velocidad_inversa_clasica}
    \mathbf{u}^\prime = - \mathbf{u}
\end{equation}

\section{Billar relativista}

Dejando de lado la teoría clásica nos trasladamos al marco de la relatividad especial, donde nos encontramos con la restricción de una velocidad máxima a la que una partícula puede moverse. En este apartado asumimos un espacio plano donde el movimiento de la partícula no se produce por fuerzas gravitacionales sino por fuerzas externas. 

\vspace{3mm}

Dado que estamos usando la relatividad especial es conveniente emplear la notación caracterísitica de esta teoría. Así definimos un cuadrivector para el momento, a partir de ahora lo llamaremos 4-momentum, que contiene tanto la energía como el momento de la partícula \cite{RelatividadBook}
%\textcolor{red}{Referencia libro - Hobson M.P., Efstathiou G.P., Lasenby A.N. - GR - An Introduction for Physicists(Cambridge,2006) }
% Referencia libro - Hobson M.P., Efstathiou G.P., Lasenby A.N. - GR - An Introduction for Physicists(Cambridge,2006)
% Pagina 119
\begin{equation}\label{eq:4-momentum}
    p^\mu = (E/c, \: \mathbf{p})
\end{equation}

donde \( E \) es la energía de la partícula medida en el sistema de referencia \( S_0 \) y \( \mathbf{p} \) es el momento medido en el sistema \( S_0 \) asociado a la partícula. Las componentes de este cuadrivector se expresan en función del factor de Lorentz de la siguiente manera

\begin{align}
    E &= \gamma m c^2 \\
    \mathbf{p} &= \gamma m \mathbf{u}
\end{align}

Con esta definición del 4-momentum es posible calcular su cuadrado y obtener su invariante asociado

\begin{equation}\label{eq:invariamente_momento_energia}
    p^\mu p_\mu \longrightarrow E^2 - p^2 c^2 = m^2c^4
\end{equation}

\vspace{3mm}

\subsection{Una dimensión}

Centrándonos ahora en una única dimension vamos a calcular como se ve modificada la velocidad tras la colisión de dos objetos. Consideremos una pared inicialmente en reposo con masa \( m_1 \) que tras la colisión obtiene una velocidad \( \bar{w} \), y una partícula con velocidad inicial \( u \) y masa \( m _2 \) que tras colisionar tiene una velocidad \( \bar{u} \). En primer lugar vamos a modificar la expresión del momento para una velocidad genérica \( v \) de la siguiente manera

\begin{align}
    \gamma &= \dfrac{1}{\sqrt{1 - \dfrac{v^2}{c^2}}} \\
    \gamma^2\left( 1 - \dfrac{u^2}{c^2} \right) &= 1 \nonumber \\
    \gamma v &= c\sqrt{\gamma^2 -1}
\end{align}

Así la ecuación \ref{eq:4-momentum} nos queda

\begin{equation}
    p^\mu = \left( E/c, \: mc\sqrt{\gamma^2 -1} \right)
\end{equation}

En nuestro caso llegamos a las expresiones

\begin{align}
    p^\mu_1 &= \left( E_1/c, 0 \right)  &\bar{p}^{\:\mu}_1 = \left( \bar{E_1}/c, \: c\sqrt{\bar{\gamma}^2_1 -1}  \right) \nonumber \\
    p^\mu_2 &= \left( E_2/c, \: c\sqrt{\gamma^2_2 -1} \right)   &\bar{p}^{\:\mu}_2 = \left( \bar{E_2}/c, \: c\sqrt{\bar{\gamma}^2_2 -1}  \right) \nonumber
\end{align}

donde \( \bar{p}^{\:\mu}_i \) son los 4-momentum tras la colisión. Dado que estamos considerando colisiones totalmente elásticas el 4-momentum se conserva, y podemos obtener la velocidad de la partícula. Entonces, el sistema de ecuaciones a resolver es

\begin{equation}
    p^\mu_1 + p^\mu_2 = \bar{p}^{\:\mu}_1 + \bar{p}^{\:\mu}_2
\end{equation}
\begin{align}
        \begin{split}
            m_1\gamma_1 + m_2 &= m_1\bar{\gamma}_1 + m_2\bar{\gamma}_2 \\[1mm]
        0 + c\sqrt{\gamma^2_2 -1} &= c\sqrt{\bar{\gamma}^2_1 -1} + c\sqrt{\bar{\gamma}^2_2 -1}
        \end{split}
\end{align}

De aquí se obtiene la velocidad de la partícula tras la colisión

\begin{align}\label{eq:velocidad_colision_1D}
    \bar{\gamma}_2 &= \dfrac{2 m_1 m_2 + \gamma_2(m_1^2 + m_2^2)}{m_1^2+m_2^2+2 m_1 m_2 \gamma_2} \\[3mm]
    \bar{u} &= \dfrac{m_2^2 - m_1^2}{m_1^2 + m_2^2 + 2 m_1 m_2 / \gamma_2}u
\end{align}

En el caso de tomar la pared con una masa \( m_1 \gg m_2 \) obtenemos el mismo resultado que en el caso clásico \ref{eq:velocidad_inversa_clasica}. Con este resultado podemos considerar ahora que la pared se encuentra en movimiento con una velocidad \( w \). Para obtener la velocidad de la partícula nos cambiamos al sistema de referencia S (mostrado en la figura \ref{fig:cambio_referencia}) donde se mueve con una velocidad igual a la del muro \( w \), de este modo la pared permanece en reposo y podemos aplicar \ref{eq:velocidad_colision_1D}. Tras ello volvemos al sistema de referencia inicial \( S_0 \).

\vspace{3mm}

En este nuevo sistema de referencia la posición y tiempo son

\begin{align}\label{eq:tiempo_sistema_muro_reposo}
    x_S &= \gamma(x - wt) \\
    t_S &= \gamma(t - \dfrac{xw}{c^2})
\end{align}

que tomando derivadas y dividiendo se obtiene como se transforma la velocidad

\begin{align}
    dx_S &= \gamma\left( dx - w dt \right) \nonumber\\
    dt_S &= \gamma \left( dt - \dfrac{w dx}{c^2} \right) \nonumber \\
    \dfrac{dx_S}{dt_S} &= u_S = \dfrac{u - w}{1 - u w/c^2} \label{eq:transformacion_velocidad_1D} 
\end{align}

\vspace{3mm}

Al colisionar con la pared se invierte el sentido según \ref{eq:velocidad_inversa_clasica}, entonces la velocidad en el sistema de referencial inicial \( S_0 \) es

\begin{align}
    \bar{u_S} &= -u_S \nonumber \\
    \dfrac{\bar{u} - w}{1 - \bar{u} w/c^2} &= -\dfrac{u - w}{1 - u w/c^2} \nonumber \\
    \bar{u} &= \dfrac{-u + 2w - u{(w/c)}^2}{1 - 2uw/c^2 + {(w/c)}^2} \label{eq:nueva_velocidad_1D}
\end{align}

\vspace{3mm}

\subsection{Dos dimensiones}

El resultado anterior permanece en consonancia con la teoría clásica que se había obtenido con anterioridad cuando \( w/c \ll 1 \). Este método para obtener la velocidad tras la colisión puede aplicarse a un espacio de dos dimensiones, donde la partícula se mueve con una velocidad \( \mathbf{u} = (u_x, u_y) \) y una pared con velocidad \( \mathbf{w} = (w_x, w_y) \). El billar a estudiar tiene una geometría rectangular compuesto por paredes verticales y horizontales, de este modo podemos hacer la suposición de que las paredes se van mover con una velocidad siempre paralela a su normal. La velocidad será \( \mathbf{w} = (w_\parallel, 0) \) ó \( \mathbf{w} = (0, w_\parallel) \)\footnote{Con esta definición se puede simplificar la notación a \( \mathbf{w} = (w, 0) \), pero se mantendrá la simbología para remarcar la dirección de la velocidad que tiene la pared.} para los muros horizontales o verticales, respectivamente. Para las componentes de la partícula también tomamos esta convención, así la velocidad va a depender de si sus componentes son paralelas o perpendiculares a la normal de la pared, \( \mathbf{u} = (u_\parallel, u_\perp) \) ó \( \mathbf{u} = (u_\perp, u_\parallel) \) según colisione con la pared vertical o horizontal, respectivamente.

\begin{figure}[H]
    \centering
    \includegraphics[scale=1.2]{images/muro_vertical.pdf}
    \caption{Diagrama de un muro vertical desplazándose hacia la partícula.}
    \label{fig:muro_vertical}
\end{figure}

La posición de la partícula en el sistema de referencia \( S_0 \) viene descrita por el vector \( \mathbf{r} \) cuyas componentes, al igual que la velocidad, serán paralelas o perpendiculares a la normal de la pared. El momento es transferido únicamente en la dirección perpendicular al muro, es decir, la única componente que se ve modificada es la paralela a la normal. 


\vspace{3mm}

La transformación de la posición y con ello de la velocidad en el sistema \( S \) es

\begin{align}\label{eq:transformacion_2d}
    r_{\parallel,S} &= \gamma\left( r_\parallel - w_\parallel t \right) \nonumber\\
    r_{\perp, S} &= r_\perp  \\
    t_S &= \gamma\left( t - \dfrac{r_\parallel w_\parallel}{c^2} \right) \nonumber
\end{align}

donde el factor de Lorentz\footnote{Es fácil ver que \( \lVert \mathbf{w} \rVert = w_\parallel \rightarrow w^2 = w_\parallel^2 \)} es \( \gamma = \dfrac{1}{\sqrt{1 - \dfrac{w^2}{c^2}}} \)

\vspace{6mm}

Derivando cada componente de \ref{eq:transformacion_2d} y dividiendo como se ha hecho anteriormente, se obtiene

\begin{align}
    \dfrac{dr_{\parallel, S}}{dt_S} = u_{\parallel,S} &= \dfrac{u_\parallel - w_\parallel}{1 - w_\parallel u_\parallel / c^2} \label{eq:velocity_change1}\\[2mm]
    \dfrac{dr_{\perp, S}}{dt_S} = u_{\perp,S} &= u_\perp\dfrac{\sqrt{1 - {(w/c)}^2}}{1 - u_\parallel w_\parallel / c^2} \label{eq:velocity_change2}
\end{align}

Tras la colisón en este sistema de referencia \( S \) con la pared en reposo, las componentes se ven modificadas de acuerdo a: \( \bar{u}_{\parallel, S} = -u_{\parallel, S} \) y \( \bar{u}_{\perp, S} = u_{\perp, S} \). Entonces volviendo al sistema de referencial inicial \( S_0 \) concluimos que las nuevas componentes de la velocidad para la partícula son

\begin{align}
    \bar{u}_\parallel &= \dfrac{-u_\parallel + 2w_\parallel - u_\parallel(w_\parallel/c)^2}{1 - 2w_\parallel u_\parallel / c^2 + (w_\parallel/c)^2} \label{eq:velocidad_paralela}\\[2mm]
    \bar{u}_\perp &= \dfrac{u_\perp\left(1 - (w_\parallel/c)^2\right)}{1 - 2w_\parallel u_\parallel / c^2 + (w_\parallel/c)^2}\label{eq:velocidad_perpendicular}
\end{align}

\section{Aceleración de Fermi}

El desarrollo de la dinámica de las partículas bajo el marco relativista muestra como la velocidad de una partícula se modifica tras cada colisión. La pregunta que surge es cómo será esa variación en la velocidad tras cada colisión.  

\vspace{3mm}

Vamos a suponer que estamos en el espacio de una dismensión con dos paredes móviles con una misma velocidad pero sentidos opuestos, esto es, si una se desplaza a la derecha la otra lo hará a la izquierda. Tras cada colisión la velocidad de la partícula cambiará de la forma

\begin{equation}\label{eq:velocidad_discreta_1D}
    u_n = u_{n-1} - 2\left( u_{n-1} - w_n \cdot (-1)^{n+1} \right)
\end{equation}

donde \( n \) es la n-ésima colisión, y \( w_n \) es el módulo de la velocidad de las paredes. Dado que la partícula colisionará con una de las paredes podemos reescribir la expresión \ref{eq:velocidad_discreta_1D} como si la partícula rebotara con la pared y volviera a moverse hacia esa pared (Otra opción sería tomar a uno de los muros en resposo dejando sin modificar la velocidad en esa colisión, así \ref{eq:velocidad_discreta_1D} se aplicaría para valores \( n \) impares con la pared que se mueve y obviando el término potencia). 

\begin{equation}\label{eq:valor_absoluto_clasico}
    \left| u_n \right| =\left| -u_{n-1} + 2w_n \right|
\end{equation}

Nos encontramos así con un mapa unidimensional (\textit{one-dimensional map} en inglés) que podemos analizar. Lo mismo ocurre para el caso relativista donde la expresión en este caso es

\begin{equation}\label{eq:valor_absoluto_relativista}
    \left| u_n \right| = \left|  \dfrac{-u_{n-1} + 2w_n - u_{n-1}(w_n/c)^2}{1 - 2w_n u_{n-1}/c^2 + (w_n/c)^2} \right|
\end{equation}

Para analizar ambos casos es conveniente usar \textit{Cobwebs} \cite{Strogatz} , las cuales son gráficas que muestran hacia donde evoluciona la función en cada iteración. Así las gráficas obtenidas para ambos casos son las figuras mostradas en \ref{fig:Cobweb_Classic_1D}

\begin{figure}[H]
    \centering
    \begin{subfigure}[b]{0.49\textwidth}
        \centering
        \includegraphics[scale=1.5]{images/Cobweb_Classic_1D_A.pdf}
        \caption{$w_n>0$}
        \label{fig:Cobweb_Classic_1D_A}
    \end{subfigure}
    \hfill
    %\hspace{0mm}
    \begin{subfigure}[b]{0.49\textwidth}
        \centering
        \includegraphics[scale=1.5]{images/Cobweb_Classic_1D_B.pdf}
        \caption{$w_n<0$}
        \label{fig:Cobweb_Classic_1D_B}
    \end{subfigure}
    \caption{Cobwebs en el caso clásico donde se ha considerado que la partícula inicia con una velocidad $u_0 > 0$ y colisionará con la pared situada a la derecha inicialmente. a) La pared se mueve en la misma dirección que la partícula. b) La pared va en sentido contrario a la partícula. En ambas imágenes se muestra en azul la función \ref{eq:valor_absoluto_clasico} y en gris \( u_n = u_{n-1} \).}
        \label{fig:Cobweb_Classic_1D}
\end{figure}

Podemos observar como ambas rectas son paralelas y la nueva velocidad que obtiene la partícula va a depender de la dirección que tome la velocidad de la pared. Así en aquellos casos donde las paredes se muevan en el mismo sentido de la partícula desacelerarán la partícula hasta llegar a una velocidad \( u_k \) donde la partícula dejará de colisionar con la pared móvil, ya que se llegará a un punto donde \( u_k < w_n \). En el caso opuesto donde las paredes siempre se dirijan hacia la partícula veremos como su velocidad aumenta sin ningún límite tras cada colisión. 

\vspace{3mm}

\begin{figure}[H]
    \centering
    \begin{subfigure}[b]{0.3\textwidth}
        \centering
        \includegraphics[scale=1.3]{images/Cobweb_Relatitivy_1D_A.pdf}
        \caption{$w_n<0$}
        \label{fig:Cobweb_Rel_1D_A}
    \end{subfigure}
    \hfill
    \begin{subfigure}[b]{0.3\textwidth}
        \centering
        \includegraphics[scale=1.3]{images/Cobweb_Relatitivy_1D_B.pdf}
        \caption{$w_n<0$}
        \label{fig:Cobweb_Rel_1D_B}
    \end{subfigure}
    \hfill
    \begin{subfigure}[b]{0.3\textwidth}
        \centering
        \includegraphics[scale=1.3]{images/Cobweb_Relatitivy_1D_C.pdf}
        \caption{$w_n>0$}
        \label{fig:Cobweb_Rel_1D_C}
    \end{subfigure}
    \caption{Cobwebs en el caso relativista donde se ha considerado \( c=1 \) y la partícula  con una velocidad inicial $u_0 > 0$ y colisionará con la pared situada a la derecha inicialmente. a), b) La pared va en sentido contrario a la partícula con distinta velocidad para cada figura. c) La pared se mueve en la misma dirección y sentido que la partícula. En todas las imágenes se muestra en azul la función \ref{eq:valor_absoluto_relativista} y en gris \( u_n = u_{n-1} \).}
        \label{fig:Cobweb_Rel_1D}
\end{figure}

Para el caso donde utilizamos la teoría relativista (simplificando las expresiones usando \( c = 1\)) obtenemos resultados similares. Cuando la velocidad de la pared tiene la misma dirección de la partícula (Figura \ref{fig:Cobweb_Rel_1D_C}) se puede ver como decrece la velocidad llegando también a una velocidad \( u_k < w_n \). El caso opuesto (Figuras \ref{fig:Cobweb_Rel_1D_A}, \ref{fig:Cobweb_Classic_1D_B}) se observa un aumento en la velocidad pero con un límite marcado por la intesección de ambas rectas, que en este caso es la velocidad de la luz,

\begin{align}
    u_n &= f(u_n) \nonumber\\
    u_n &= -\dfrac{-u_{n-1} + 2w_n - u_{n-1}(w_n/c)^2}{1 - 2w_n u_{n-1}/c^2 + (w_n/c)^2}\label{eq:Solucion_cobweb} \\
    0 &= \dfrac{w_n\left( 2u_n^2-2c^2 \right)}{c^2} \nonumber\\
    u_n &= \pm c \label{eq:Solucion_cobweb,valor}
\end{align}

\vspace{3mm}

La solución encontrada son dos puntos fijos pero debemos quedarnos con la positiva, ya que la velocidad que obtenemos debe ser siempre positiva debido al valor absoluto. Además ese valor negativo en \ref{eq:Solucion_cobweb} implica usar el trozo de la función situada en el primer cuadrante. Podemos profundizar más y ver qué estabilidad tiene este punto

\begin{equation}
    \dfrac{du_n}{du} = f'(u_n) = \dfrac{1 - 2w^2/c^2 + w^4/c^4}{1 + 4/c^4\left( u^2w^2-uw^3 \right) + 4/c^2\left( u^2w^2-uw \right) + w^4/c^4}
\end{equation}

\begin{figure}[H]
    \centering
    \includegraphics[scale=0.9]{images/Estabilidad_Cobweb.pdf}
    \caption{Diagrama de la derivada de $u_n$ evaluada en el punto fijo en función de la velocidad de la pared. En rojo la estabilidad del punto.}
    \label{fig:estabilidad_cobweb}
\end{figure}

\vspace{3mm}

Para ver la estabilidad del punto se debe analizar si \( \left| f'(c) \right| \) es mayor o menor a la unidad \cite{Strogatz}. Usando la figura \ref{fig:estabilidad_cobweb} podemos ver como para valores \( w_n < 0 \) se cumple \( \left| f'(c) \right| < 1 \), siendo estables, y lo opuesto ocurre para \( w_n > 0 \) conviertiéndose en inestables.

\vspace{3mm}

Estos resultados evidencian el caso de estudio de este trabajo: La aceleración de Fermi. En capítulos próximos veremos que relación tiene con la distancia o área de los billares considerados.

\section{Colisión inelástica}

Hasta ahora se ha supuesto que la colisión es totalmente elástica y no existe pérdida de energía tras cada colisión, lo que ocasiona que pueda existir la aceleración de Fermi. Vamos a suponer que tras cada colisión la velocidad de la partícula, en un sistema donde la pared está en reposo, se ve modificada de acuerdo a 

\begin{equation}
    \bar{u} = -\epsilon u \qquad \epsilon \in \left[ 0, 1 \right]
\end{equation}

donde \( \epsilon \) es el factor de restitución. Para obtener la solución en el caso relativista podemos usar de nuevo las expresiones \ref{eq:velocity_change1} y \ref{eq:velocity_change2}, pero teniendo en cuenta que al igual que antes la única componente de la velocidad que se ve modificada es la paralela a la normal, manteniéndose constante la perpendicular. 

\vspace{3mm}

Entonces se tiene

\begin{equation}
    \bar{u}_\parallel = \dfrac{-\epsilon u_\parallel + w_\parallel (1 + \epsilon) -  u_\parallel (w_\parallel/c)^2}{1 - (1 + \epsilon)u_\parallel w_\parallel/c^2 + \epsilon (w_\parallel/c)^2}
\end{equation}

que si usamos \( \epsilon = 1 \) como en una colisión totalmente elástica obtenemos la expresión \ref{eq:velocidad_paralela}

\vspace{3mm}

De nuevo nos encontramos con una función discreta que sólo se modifica cuando existe una colisión. Así podemos emplear el mismo análisis para el caso elástico y ver si aparece la aceleración de Fermi. Considerando que siempre impacta contra la misma pared podemos obtener la velocidad máxima que podrá adquirir la partícula

\begin{align}\label{eq:Solucion_cobweb_inelastica}
    u_n &= g(u_n) \nonumber\\
    u_n &= -\dfrac{-\epsilon u_{n-1} + w_n(1+\epsilon) -  u_{n-1}(w_n/c)^2}{1 - (1+\epsilon)w_n u_{n-1}/c^2 + \epsilon(w_n/c)^2} \\
    0 &= u_n^2\left[ -(1+\epsilon)w/c^2 \right] + u\left[ 1 - \epsilon + (\epsilon - 1)w/c^2 \right] + w(1+\epsilon) \nonumber\\
    u_n &= \dfrac{-\left[ 1 -  (\epsilon - 1)w/c^2 \right] \pm \sqrt{\left[ 1 - \epsilon (\epsilon - 1)w/c^2 \right]^2 + 4(1+\epsilon)^2(w/c)^2}}{2(1+\epsilon)w/c^2} \label{eq:Solucion_cobweb_inelastica_valor}
\end{align}

Si volvemos al caso elástico con \( \epsilon = 1 \) obtenemos el mismo resultado que en \ref{eq:Solucion_cobweb,valor}, por lo tanto debemos quedarnos con la solución correspondiente a la suma de los términos del numerador. 

\begin{figure}[H]
    \centering
    \includegraphics[scale=0.8]{images/Simulaciones/Relatividad_1D/Plano_fases.pdf}
    \caption{Velocidad máxima de la partícula para las distintas configuraciones del coeficiente de restitución y de la pared.}
    \label{fig:plano_fases_coef_wall}
\end{figure}

\begin{figure}[H]
    \centering
    \begin{subfigure}[b]{0.49\textwidth}
        \centering
        \includegraphics[scale=1.5]{images/Cobweb_Relativity_inelastic_e05_w-01.pdf}
        \caption{$\varepsilon = 0.5$}
        \label{fig:Cobweb_inelastics_A}
    \end{subfigure}
    \hfill
    %\hspace{0mm}
    \begin{subfigure}[b]{0.49\textwidth}
        \centering
        \includegraphics[scale=1.5]{images/Cobweb_Relativity_inelastic_e08_w-01.pdf}
        \caption{$\varepsilon = 0.8$}
        \label{fig:Cobweb_inelastics_B}
    \end{subfigure}
    \caption{Cobwebs para el caso con colisiones relativistas inelásticas considerando $c=1$. a) La partícula inicia con una velocidad cercana a la de la luz. b) La velocidad inicia con una velocidad baja. }
        \label{fig:Cobweb_inelastics}
\end{figure}

Los diagramas \ref{fig:Cobweb_inelastics} muestran como el valor dado por \ref{eq:Solucion_cobweb_inelastica_valor} siempre es atractivo, es decir, cualquier condición inicial va a terminar con la velocidad máxima dada.

%\end{document}