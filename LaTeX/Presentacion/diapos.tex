\documentclass{beamer}
\usepackage[spanish]{babel}
\usetheme{Antibes}
\usepackage{tikz}
\usepackage{xcolor}
\usepackage{amsmath,amssymb}
\usepackage{enumitem}
\usepackage{caption}
\usepackage[font={color=red,bf},labelfont={it}]{caption}
\captionsetup{font=scriptsize,labelfont=scriptsize, figurename=Fig., labelfont={color=verdeodi}, justification=centering}


\usepackage[backend=biber,
style=numeric,
maxbibnames=10,
sorting=none,
citestyle=numeric-comp]{biblatex}
\addbibresource{referenciasPPT.bib}

% ---------- DISEÑO ------------------
\definecolor{verdeodi}{RGB}{53,113,105} %%%%%%%

\setbeamercolor*{structure}{bg=red,fg=verdeodi}

\setbeamercolor*{palette primary}{use=structure,fg=white,bg=structure.fg}
\setbeamercolor*{palette secondary}{use=structure,fg=white,bg=structure.fg!75}
\setbeamercolor*{palette tertiary}{use=structure,fg=white,bg=verdeodi}
\setbeamercolor*{palette quaternary}{fg=white,bg=verdeodi!50!black}

\setbeamercolor{section in toc}{fg=black,bg=white}
\setbeamercolor{alerted text}{use=structure,fg=structure.fg!50!verdeodi!80!black}

\setbeamercolor{titlelike}{parent=palette primary,fg=structure.fg!50!black}
\setbeamercolor{frametitle}{bg=verdeodi!50,fg=white}

\setbeamercolor*{titlelike}{parent=palette primary}

\setbeamertemplate{caption}[numbered]
% ---------- O ------------------

\title{Estudio de la presencia de la aceleración de Fermi en billares rectangulares}
\subtitle{Grado en Física\\[1mm]Trabajo fin de grado}
\author{Borja Sánchez González}
\institute{Universidad Nacional de Educación a Distancia}
\date{\empty}

\AtBeginSection[]
{
  \begin{frame}
    \frametitle{Índice}
    \tableofcontents[currentsection]
  \end{frame}
}

\expandafter\def\expandafter\insertshorttitle\expandafter{%
  \insertshorttitle\hfill%
  \insertframenumber\,/\,\inserttotalframenumber}

\begin{document}
% -------- TITULO -----------
\begin{frame}
    \titlepage%
\end{frame}
% -------- O -----------

% -------- INDICE -----------
\begin{frame}
    \frametitle{Índice}
    \tableofcontents
\end{frame}
% -------- O -----------

% -------- INTRODUCCION-----------
\section{Introducción}

\subsection{¿Qué es un billar?}

%\begin{frame}
%    \frametitle[prueb1]{Concepto}
%    \begin{columns}
%        \begin{column}{0.5\textwidth}
%            \begin{figure}
%                \includegraphics[scale=1]{images/bunimovich.pdf}    
%                \caption{Bunimovich}
%            \end{figure}  
%            
%        \end{column}
%    
%        \begin{column}{0.5\textwidth}
%            \begin{figure}
%                \includegraphics[scale=1]{images/sinai.pdf}
%                \caption{Sinai}
%            \end{figure}
%        \end{column}
%    \end{columns}
%\end{frame}

\begin{frame}
    \frametitle[prueb1]{Concepto}
    \begin{align*}
        H = \dfrac{p^2}{2m} + V & & V = \begin{cases}
            0 \text{ en el interior de } \Omega \\
            \infty \text{ en el exterior de } \Omega
        \end{cases}
    \end{align*}
    \begin{figure}
        \centering
        %\includegraphics[scale=0.55]{../images/billar_generico.pdf}
        \includegraphics[scale=1]{images/bunimovich.pdf}
    \end{figure}
\end{frame}

\subsection{¿Qué es la aceleración de Fermi?}

\begin{frame}
    \frametitle[prueb1]{Concepto}
    \begin{columns}
    \begin{column}{0.5\textwidth}
        \textbf{Aceleración de Fermi}\\
        Gran aumento de energía que adquiere una partícula tras varias interacciones.
    \end{column}
    \begin{column}{0.5\textwidth}
        \begin{figure}
            %[trim={left bottom right top},clip]
            %\includegraphics[scale=2.8, trim={0cm 0cm 0cm 0.5cm}]{images/aceleracion.pdf}    
            \includegraphics[scale=0.1]{images/Enrico_Fermi_1943-49.jpeg}
            \caption{Enrico Fermi}  
        \end{figure}    
    \end{column}

    \end{columns}
\end{frame}

%\begin{frame}
%    \begin{columns}
%        \begin{column}{0.5\textwidth}
%            \begin{figure}
%                \centering
%                \includegraphics[scale=0.1]{images/Enrico_Fermi_1943-49.jpeg}  
%                \caption{Enrico Fermi}  
%            \end{figure}            
%        \end{column}
%
%        \begin{column}{0.5\textwidth}
%            \begin{figure}
%                \centering
%                \includegraphics[scale=2.5]{images/Fermi-ulam.pdf}  
%            \end{figure}            
%        \end{column}
%    \end{columns}
%\end{frame}


% -------- O -----------

% -------- DINAMICA -----------
\section{Partícula en un billar}
\subsection{Ecuaciones de movimiento}

\begin{frame}
    \frametitle[prueb1]{Dinámica general}
    \begin{columns}
        \begin{column}{0.5\textwidth}
            \begin{figure}
                \centering
                \includegraphics[scale=1.2]{../images/elastic_collision.pdf}
            \end{figure}         
        \end{column}
        \begin{column}{0.5\textwidth}
            \centering
            \textbf{Interior}
            \begin{align*}
                \dot{q} = \dfrac{\partial H}{\partial p} &\rightarrow q(t) = ut \\
                \dot{p} = -\dfrac{\partial H}{\partial q} &\rightarrow p(t) = cte = mu
            \end{align*}\\
            \vspace{0.6cm}
            \textbf{Frontera}
            \begin{align*}
                \mathbf{p}' &= \mathbf{p} - 2\mathbf{n}(\mathbf{p} \cdot \mathbf{n}) \\
                p'_\parallel &= -p_\parallel
            \end{align*}
        \end{column}
    \end{columns}
\end{frame}

%\begin{frame}
%    \frametitle[prueb1]{Dinámica}
%    \begin{columns}
%        \begin{column}{0.5\textwidth}
%            \begin{figure}
%                \centering
%                \includegraphics[scale=1.2]{../images/elastic_collision.pdf}
%            \end{figure}         
%        \end{column}
%        \begin{column}{0.5\textwidth}
%            \centering
%            Pared en reposo
%            \begin{equation*}
%                \mathbf{p}' = \mathbf{p} - 2\mathbf{n}(\mathbf{p} \cdot \mathbf{n})
%            \end{equation*}\\
%            \vspace{1cm}
%            Pared en movimiento
%            \begin{equation*}
%               \mathbf{u}' = \mathbf{u} - 2\mathbf{n}\left[(\mathbf{u} - \mathbf{w}) \cdot \mathbf{n}\right]
%            \end{equation*}
%        \end{column}
%    \end{columns}
%\end{frame}

\begin{frame}
    \frametitle[prueb1]{Relativista}
    \begin{columns}
        \hspace{0.5cm}
        \begin{column}{0.5\textwidth}
            \vspace{1.5mm}
            \begin{figure}
                \centering
                \includegraphics[scale=1]{../images/cambio_sistema_referencia.pdf}
                \caption{Sistemas de referencia}
            \end{figure}         
        \end{column}
        %\hspace{0.5cm}
        \begin{column}{0.5\textwidth}
            \begin{figure}
                \centering
                \includegraphics[scale=1.15]{../images/muro_vertical.pdf}
                \caption{Componentes de velocidad}
            \end{figure}
        \end{column}
    \end{columns}
\end{frame}

\begin{frame}[t]
    \frametitle[prueb1]{Relativista}
    \begin{columns}
        \begin{column}{0.5\textwidth}
            \begin{block}{Paso 1}
                \begin{align*}
                    r_{\parallel,S} &= \gamma\left( r_\parallel - w_\parallel t \right) \nonumber\\
                    r_{\perp, S} &= r_\perp  \\
                    t_S &= \gamma\left( t - \dfrac{r_\parallel w_\parallel}{c^2} \right) \nonumber
                \end{align*}
                \vspace{5.5mm}
            \end{block}
        \end{column}
        \begin{column}{0.5\textwidth}
        \end{column}
    \end{columns}
\end{frame}

\begin{frame}[t]
    \frametitle[prueb1]{Relativista}
    \begin{columns}
        \begin{column}{0.5\textwidth}
            \begin{block}{Paso 1}
                \begin{align*}
                    r_{\parallel,S} &= \gamma\left( r_\parallel - w_\parallel t \right) \nonumber\\
                    r_{\perp, S} &= r_\perp  \\
                    t_S &= \gamma\left( t - \dfrac{r_\parallel w_\parallel}{c^2} \right) \nonumber
                \end{align*}
                \vspace{5.5mm}
            \end{block}
        \end{column}
        \begin{column}{0.5\textwidth}
            \begin{block}{Paso 2}
                \begin{align*}
                    u_{\parallel,S} &= \dfrac{u_\parallel - w_\parallel}{1 - w_\parallel u_\parallel / c^2} \\[5mm]
                    u_{\perp,S} &= u_\perp\dfrac{\sqrt{1 - {(w_\parallel/c)}^2}}{1 - u_\parallel w_\parallel / c^2} 
                \end{align*}
            \end{block}
        \end{column}
    \end{columns}
    \begin{tikzpicture}[line/.style={->,shorten >=0.4cm,shorten <=0.4cm}, very thick]
        \draw[] (0,0) -- (0, 0);
        \path[verdeodi!50!black,bend right,line] (2,-0) edge (8.7,-0);
        \node at (5.4, -0.6) {$\frac{dr}{dt}$};
    \end{tikzpicture}
\end{frame}

\begin{frame}[t]
    \frametitle[prueb1]{Relativista}
    \begin{columns}
        \begin{column}{0.5\textwidth}
            \begin{block}{Paso 3}
                \begin{align*}
                    \bar{u}_{\parallel, S} &= -u_{\parallel, S} \\[9.5mm]
                    \bar{u}_{\perp, S} &= u_{\perp, S}
                \end{align*}
                \vspace{2.2mm}
            \end{block}
        \end{column}
        \begin{column}{0.5\textwidth}
        \end{column}
    \end{columns}
\end{frame}

\begin{frame}[t]
    \frametitle[prueb1]{Relativista}
    \begin{columns}
        \begin{column}{0.5\textwidth}
            \begin{block}{Paso 3}
                \begin{align*}
                    \bar{u}_{\parallel, S} &= -u_{\parallel, S} \\[9.5mm]
                    \bar{u}_{\perp, S} &= u_{\perp, S}
                \end{align*}
                \vspace{2.2mm}
            \end{block}
        \end{column}
        \begin{column}{0.5\textwidth}
            \begin{block}{Resultado}
                \begin{align*}
                    \bar{u}_\parallel &= \dfrac{-u_\parallel + 2w_\parallel - u_\parallel(w_\parallel/c)^2}{1 - 2w_\parallel u_\parallel / c^2 + (w_\parallel/c)^2} \\[5mm]
                    \bar{u}_\perp &= \dfrac{u_\perp\left(1 - (w_\parallel/c)^2\right)}{1 - 2w_\parallel u_\parallel / c^2 + (w_\parallel/c)^2}
                \end{align*}
            \end{block}
        \end{column}
    \end{columns}
\end{frame}

\begin{frame}
    \frametitle[prueb1]{Discretización}
    \begin{equation*}
        \left| u_n \right| = \left|  \dfrac{-u_{n-1} + 2w_n - u_{n-1}(w_n/c)^2}{1 - 2w_n u_{n-1}/c^2 + (w_n/c)^2} \right|
    \end{equation*}
    \begin{figure}
        \includegraphics[scale=2]{images/enfrentadas.pdf}
    \end{figure}
\end{frame}

\begin{frame}
    \frametitle[prueb1]{Discretización}
    \begin{equation*}
        \left| u_n \right| = \left|  \dfrac{-u_{n-1} + 2w_n - u_{n-1}(w_n/c)^2}{1 - 2w_n u_{n-1}/c^2 + (w_n/c)^2} \right|
    \end{equation*}
    \begin{tikzpicture}[line/.style={->,shorten >=0.4cm,shorten <=0.4cm}, very thick]
        \draw[] (0,0) -- (0, 0);
        \draw[-stealth] (5.5, 0) -- (5.5, -2);
    \end{tikzpicture}
    \begin{equation*}
        u_n = -\dfrac{-u_{n-1} + 2w_n - u_{n-1}(w_n/c)^2}{1 - 2w_n u_{n-1}/c^2 + (w_n/c)^2}
    \end{equation*}
\end{frame}

\begin{frame}
    \frametitle[prueb1]{Cobwebs}{}
    \centering
    \textbf{Clásico}
    \begin{columns}
        \begin{column}{0.5\textwidth}
            \vspace{0.5cm}
            \begin{figure}
                \centering
                %[trim={left bottom right top},clip]
                \includegraphics[scale=1, trim={0.5cm 0cm 0cm 0cm}]{../images/Cobweb_Classic_1D_A.pdf}
                \caption{Partícula y pared moviéndose en el mismo sentido ($w>0$).}
            \end{figure}  
        \end{column}
        \begin{column}{0.5\textwidth}
            \begin{figure}
                \centering
                \includegraphics[scale=1]{../images/Cobweb_Classic_1D_B.pdf}
                \caption{Partícula y pared moviéndose en sentidos opuestos ($w<0$).}
            \end{figure}  
        \end{column}
    \end{columns}
    \begin{equation*}
             \color{gray} u_n \color{gray} =  u_{n-1} \qquad\qquad \color{blue} f(u_n) \color{blue} =  u_{n-1} - 2w_n
    \end{equation*}
\end{frame}

\begin{frame}
    \frametitle[prueb1]{Cobwebs}{}
    \centering
    \textbf{Relativista}
    \begin{columns}
        \begin{column}{0.5\textwidth}
            \begin{figure}
                \centering
                \includegraphics[scale=1, trim={0.5cm 0cm 0cm 0cm}]{../images/Cobweb_Relatitivy_1D_C.pdf}
                \caption{Partícula y pared moviéndose en el mismo sentido  ($w>0$).}
            \end{figure}  
        \end{column}
        \begin{column}{0.5\textwidth}
            \begin{figure}
                \centering
                \includegraphics[scale=1]{../images/Cobweb_Relatitivy_1D_B.pdf}
                \caption{Partícula y pared moviéndose en sentidos opuestos ($w<0$).}
            \end{figure}  
        \end{column}
    \end{columns}
    \begin{equation*}
        \color{gray} u_n \color{gray} =  u_{n-1} \qquad\qquad \color{blue} \scalebox{0.8}{ $f(u_n)$} \color{blue} = \scalebox{0.8}{$-\dfrac{-u_{n-1} + 2w_n - u_{n-1}(w_n/c)^2}{1 - 2w_n u_{n-1}/c^2 + (w_n/c)^2}$}
    \end{equation*}
\end{frame}

\begin{frame}
    \frametitle[prueb1]{Aproximación inelástica}
    \begin{equation*}
        \bar{u} = -\epsilon u \qquad \epsilon \in \left[ 0, 1 \right]
    \end{equation*}
    \begin{tikzpicture}[line/.style={->,shorten >=0.4cm,shorten <=0.4cm}, very thick]
        \draw[] (0,0) -- (0, 0);
        \draw[-stealth] (5.5, 0) -- (5.5, -2);
    \end{tikzpicture}
    \begin{equation*}
        \bar{u}_\parallel = \dfrac{-\epsilon u_\parallel + w_\parallel (1 + \epsilon) -  u_\parallel (w_\parallel/c)^2}{1 - (1 + \epsilon)u_\parallel w_\parallel/c^2 + \epsilon (w_\parallel/c)^2}
    \end{equation*}
\end{frame}

\begin{frame}
    \frametitle[prueb1]{Inelástico} 
        \centering
        \vspace{-0.7cm}
            \begin{equation*}
                \scalebox{0.6}{$u_n = \dfrac{-\left[ 1 -  (\epsilon - 1)w/c^2 \right] + \sqrt{\left[ 1 - \epsilon (\epsilon - 1)w/c^2 \right]^2 + 4(1+\epsilon)^2(w/c)^2}}{2(1+\epsilon)w/c^2}$ }
            \end{equation*}
            \vspace{-1cm}
            \begin{figure}
                \centering
                \includegraphics[scale=0.45, trim={0cm 0cm 1cm 0.8cm}, clip]{../images/Simulaciones/Relatividad_1D/Plano_fases.pdf}
                \caption{Velocidades máximas de la partícula.}
            \end{figure}
\end{frame}

% -------- O -----------

% -------- SIMULACIONES -----------
\section{Simulaciones}
\subsection{Resultados numéricos}

%\begin{frame}[t]
%    \frametitle[prueb1]{Simulación}
%    \begin{columns}
%        \hspace{-1cm}
%        \begin{column}{0.5\textwidth}
%            \begin{figure}
%                \centering
%                \includegraphics[scale=0.25]{../images/billar_generico_3.pdf}
%            \end{figure}  
%        \end{column}
%        \begin{column}{0.5\textwidth}
%            \begin{figure}
%                \centering
%                \includegraphics[scale=0.25]{../images/Billiard_45.pdf}
%            \end{figure}  
%        \end{column}
%    \end{columns}
%\end{frame}

\begin{frame}
    \frametitle[prueb1]{Simulación}
    \begin{columns}
        \begin{column}{0.5\textwidth}
            \begin{itemize}
                \item Código desarrollado en Python
                \item 10.000 partículas
                \item \( v_0 \) pseudoaleatoria
            \end{itemize}
        \end{column}
        \hspace{-1.3cm}
        \begin{column}{0.5\textwidth}
            \begin{figure}
                \centering
                \includegraphics[scale=0.3, trim={3cm 2cm 2cm 2cm}]{../images/Billiard_45.pdf}
                \caption{Trayectorias simuladas para una partícula.}
            \end{figure}  
        \end{column}
    \end{columns}
\end{frame}

\begin{frame}[t]
    \frametitle[prueb1]{Unidimensional}
    \begin{columns}
        \hspace{-0.7cm}
        \begin{column}{0.5\textwidth}
            \begin{figure}
                \centering
                \hspace{1.cm}
                \textbf{Caso clásico}\par\medskip
                \vspace{-0.2cm}
                \includegraphics[scale=0.4, trim={0.5cm 0cm 1.5cm 0.9cm}, clip]{../images/Simulaciones/Clasico_1D/1D_B-N1000.pdf}
                \caption{Velocidad media frente a colisión}
            \end{figure}         
        \end{column}
        \begin{column}{0.5\textwidth}
            \begin{figure}
                \centering 
                \hspace{0.4cm}
                \textbf{Caso inelástico \( \epsilon = 0.7 \)}\par\medskip
                \vspace{-0.23cm}
                \includegraphics[scale=0.4, trim={1cm 0cm 2cm 1cm}, clip]{../images/Simulaciones/Inelastico_1D/1D_D-N1000.pdf}
                \captionof{figure}{Velocidad media frente a colisión.}
            \end{figure}
        \end{column}
    \end{columns}
\end{frame}

\begin{frame}[t]
    \frametitle[prueb1]{Unidimensional}
    \begin{columns}
        \hspace{-0.0cm}
        \begin{column}{0.5\textwidth}
            \vspace{0.5cm}
            \begin{figure}
                \centering
                \hspace{.5cm}
                \textbf{Caso relativista}\par\medskip
                \includegraphics[scale=0.4, trim={1cm 0cm 0cm 1cm}, clip]{../images/Simulaciones/Relatividad_1D/1D_C2-N1000.pdf}
                \caption{Velocidad media frente a colisión.}
            \end{figure}         
        \end{column}
        \begin{column}{0.5\textwidth}
            \begin{align*}
                f(n) &= \delta - \gamma \exp\left[ -\beta \left( \dfrac{n}{n_c} \right)^\alpha \right]
            \end{align*}
            \begin{itemize}
                \item \( \delta \) : Velocidad máxima 
                \item \( \beta \) : Tasa repetición de sucesos
                \item \( n_c \) : Número crítico de sucesos
                \item \( \gamma, \alpha \) : Parámetros
            \end{itemize}
        \end{column}
    \end{columns}
\end{frame}

\begin{frame}%[t]
    \frametitle[prueb1]{Unidimensional}
    \begin{columns}
        \hspace{0.5cm}
    \begin{column}{0.5\textwidth}
        \begin{figure}
            \centering
            \includegraphics[scale=0.35, trim={1cm 0cm 0cm 0cm}, clip]{../images/Simulaciones/Relatividad_1D/ajuste_parametros_W-001.pdf}
            \caption{Ajuste de parámetros para un mismo billar.}
        \end{figure}         
    \end{column}
    \begin{column}{0.5\textwidth}
        \vspace{1cm}
        \begin{equation*}
        \begin{cases}
             \begin{aligned}
                v_0 &\in [0.001; 0.9]c \\[0.5cm]
                w_r &= -0.01c \\ 
                w_l &= 0
            \end{aligned}
        \end{cases}
    \end{equation*}
             \vspace{1cm}
             \begin{equation*}
                \beta \propto K \dfrac{v^2_{rel}}{w} = K\dfrac{(v_0 - w)^2}{w}
            \end{equation*}
    \end{column}
    \end{columns}
\end{frame}

\begin{frame}%[t]
    \frametitle[prueb1]{Unidimensional}
    \begin{columns}
        \hspace{0.5cm}
        \begin{column}{0.5\textwidth}
            \begin{figure}
                \centering
                \includegraphics[scale=0.35, trim={1cm 0cm 0cm 0cm}, clip]{../images/Simulaciones/Relatividad_1D/ajuste_parametros_V-02.pdf}
                \caption{Ajuste de parámetros para una misma velocidad inicial.}  
            \end{figure}         
        \end{column}
    \begin{column}{0.5\textwidth}
        \begin{equation*}
            \begin{cases}
                 \begin{aligned}
                    w &\in [0.001; 0.9]c \\ 
                    v_0 &= 0.2c
                \end{aligned}
            \end{cases}
        \end{equation*}
         \vspace{1cm}
         \begin{equation*}
            \beta = \lim_{w\rightarrow 0} \dfrac{v^2_{rel}}{w}g(w) = 0
        \end{equation*}
    \end{column}
    \end{columns}
\end{frame}

\begin{frame}
    \frametitle[prueb1]{Bidimensional}{}
    \centering
    \textbf{Clásico}
    \vspace{-0.5cm}
    \begin{columns}
        \hspace{-0.cm}
        \begin{column}{0.5\textwidth}
            \begin{figure}
                \centering
                \includegraphics[scale=0.4, trim={0.5cm 0cm 0cm 0.5cm},clip]{../images/Simulaciones/Clasico_2D/2D_B-N1000.pdf}
                \caption{Velocidad media frente a colisión.}
            \end{figure}         
        \end{column}
        \begin{column}{0.5\textwidth}
            \begin{figure}
                \centering 
                \vspace{0.1cm}
                \includegraphics[scale=0.4, trim={1cm 0cm 0cm 0cm},clip]{../images/Simulaciones/Clasico_2D/2D_B-N1000_Area.pdf}
                \caption{Evolución del área en cada colisión.}
            \end{figure}
        \end{column}
    \end{columns}
\end{frame}

\begin{frame}
    \frametitle[prueb1]{Bidimensional}{}
    \centering
    \textbf{Relativista}
    \vspace{-0.5cm}
    \begin{columns}
        \hspace{-0.cm}
        \begin{column}{0.5\textwidth}
            \begin{figure}
                \centering
                \includegraphics[scale=0.4, trim={0.5cm 0cm 0cm 0.5cm},clip]{../images/Simulaciones/Relatividad_2D/2D_BR-N1000.pdf}
                \caption{Velocidad media frente a colisión.}
            \end{figure}         
        \end{column}
        \begin{column}{0.5\textwidth}
            \begin{figure}
                \centering 
                \captionsetup{justification=centering}
                \vspace{0.4cm}
                \includegraphics[scale=0.4, trim={0.8cm 0cm 0cm 0.5cm},clip]{../images/Simulaciones/Relatividad_2D/2D_BR-N1000_Area.pdf}
                \caption{Evolución del área en cada colisión.}
            \end{figure}
        \end{column}
    \end{columns}
\end{frame}

\begin{frame}[t]
    \frametitle[prueb1]{Bidimensional}{}
    \centering
    \textbf{Inelástico}
    \vspace{-0.5cm}
    \begin{columns}
        \begin{column}{0.5\textwidth}
            \begin{figure}
                \centering
                \includegraphics[scale=0.35]{../images/Simulaciones/Inelastico_2D/componentes.pdf}
                \caption{Velocidad media frente a colisión. \( \epsilon = 0.99 \)}  
            \end{figure}         
        \end{column}
    \begin{column}{0.5\textwidth}
        \begin{align*}
            u_{1D} &\leq u_{2D} \nonumber\\
            u_{\parallel} &\leq \sqrt{u_\parallel^2 + u_\perp^2}\\
             0 &\leq u_\perp^2 \nonumber
        \end{align*}
    \end{column}
    \end{columns}
\end{frame}

% -------- O -----------

% -------- CONCLUSIONES -----------
\section{Conclusiones}

\begin{frame}
    \frametitle[prueb1]{Conclusiones}
    \begin{itemize}
        \item La aceleración de Fermi aparece en billares cuya distancia/área se reduce.
        \item En ambas dimensiones la velocidad límite disminuye en gran medida con un ligero cambio en el coeficiente de restitución.
        \item Los resultados se ajustan correctamente a una distribución acumulativa.
    \end{itemize}
\end{frame}

% -------- O -----------

% -------- BIBLIOGRAFÍA -----------
%\section{Bibliografía}

\begin{frame}
    \frametitle[prueb1]{Bibliografía}
    \begin{enumerate}[topsep=0pt,itemsep=4pt,partopsep=1ex,parsep=1ex]
        \item E. Fermi. “On the origin of Cosmic Radiation”. En: (1949).
        \item L.D. Pustyl’nikov. “On a problem of Ulam”. En: (1983).
        \item R.S. Pinto y P.S. Letelier. “Fermi Acceleration in driven relativistic billiards”. En: (2011)
        \item S. Strogatz. “Nonlinear dynamics and Chaos”. En: (1994).
        \item \url{https://github.com/Xatkor/TFG_Codigo}
       %\item https://github.com/Xatkor/TFG_Codigo
    \end{enumerate}
\end{frame}

% -------- O -----------

\section{} 

\begin{frame}
    \titlepage%
    \centering
    %\vspace{-1cm}
    \Large\textbf{¡Muchas gracias!}
\end{frame}

\end{document}

