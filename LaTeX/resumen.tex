\chapter*{Resumen}
\addcontentsline{toc}{chapter}{Resumen}
\selectlanguage{spanish}

Se realiza el estudio tanto para una dimensión como en dos dimensiones de billares rectangulares móviles dentro del marco clásico y relativista. La idea principal de este trabajo es buscar la relación entre la aparición de la aceleracion de Fermi y la tendencia de las paredes del billar a aumentar o disminuir la distancia entre ellas. En el caso unidimensional sería la distancia entre ellas, pero en el bidimensional se tendría en cuenta el área del billar.

\vspace{5cm}

\selectlanguage{english}

{\let\clearpage\relax\chapter*{Abstract}}
\addcontentsline{toc}{chapter}{Abstract}

The study is carried out for both one and two dimensions of driven rectangular billiards in the classical and relativistic framework. The main idea of this work is to find the relation between the appearance of the Fermi acceleration and the tendency of the billiard walls to increase or decrease the distance between them. In the one-dimensional case it would be the distance between them, but in the two-dimensional case the area of the billiard would be taken into account.

\selectlanguage{spanish}

