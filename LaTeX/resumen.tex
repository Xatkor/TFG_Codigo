\chapter*{Resumen}
\addcontentsline{toc}{chapter}{Resumen}
\selectlanguage{spanish}

En el marco clásico y relativista, se analiza la presencia de la aceleración de Fermi en billares rectangulares cuyas paredes se desplazan a velocidad constante, tanto en una como en dos dimensiones. Con ello se busca establecer una relación entre esta aceleración y la tendencia de la distancia entre las paredes o el área del billar a aumentar o disminuir, dependiendo de la dimensión. Además, se realiza un estudio dirigido a suprimir dicha aceleración en los casos en que se manifieste la aceleración.

\vspace{5cm}

\selectlanguage{english}

{\let\clearpage\relax\chapter*{Abstract}}
\addcontentsline{toc}{chapter}{Abstract}

Within the classical and relativistic framework, the presence of Fermi acceleration has been analyzed in rectangular billiards with walls moving at a constant velocity, in both one and two dimensions. The aim is to establish a relation between this acceleration and the tendency for the distance between the walls or the area of the billiard to either increase or decrease, depending on the dimension. Additionally, a study has been developed aimed at suppressing the acceleration.






\selectlanguage{spanish}

