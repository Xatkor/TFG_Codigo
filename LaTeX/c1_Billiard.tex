\documentclass[11pt, spanish]{book}
\usepackage[T1]{fontenc}
\usepackage[utf8]{inputenc}
\usepackage[a4paper]{geometry}
\geometry{verbose,tmargin=2cm,bmargin=2cm,lmargin=2cm,rmargin=1.5cm}
\setlength{\parindent}{0\paperheight}
\usepackage{color}
%\usepackage{xcolor}
\usepackage[dvipsnames]{xcolor}
\usepackage{amsmath}
\usepackage{graphicx}
\usepackage{esint}
\usepackage{colortbl}
\usepackage{babel}
\usepackage{enumitem}
\usepackage{amssymb, amsmath}
\usepackage{tikz}
\usepackage{booktabs}
\usepackage{subcaption}
\usetikzlibrary{arrows,decorations.markings,plotmarks}
\selectlanguage{spanish}
\usepackage{caption}
\usepackage{svg}
\usepackage{fancybox}
\usepackage{enumitem}
\definecolor{verdeodi}{RGB}{53,113,105} %%%%%%%
\usepackage[font={color=verdeodi,it},figurename=Fig.,labelfont={it}]{caption}
\usepackage[most]{tcolorbox}
\usepackage{float}
\newtcolorbox{fancybox}[1][]{ %%%%%%%
enhanced,
boxrule=1pt,arc=8pt,boxsep=0pt,
left=0.8em,right=0.8em,top=1ex,bottom=1ex,colback=verdeodi!20,colframe=verdeodi,#1}
\usepackage[Glenn]{fncychap}
\newcommand{\stylecolor}{\color{verdeodi!90!black}} % choose the color
\ChNameVar{\stylecolor\Large\fontfamily{put}\selectfont}

\begin{document}

\chapter{Partícula en un billar}

Una partícula contenida en un billar puede definirse como una partícula libre sometida a un potencial infinito

\begin{equation}
    H = \dfrac{p^2}{2m} + V
\end{equation}

donde \( V \) se puede definir de manera general como

\begin{equation}
    V = \begin{cases}
        0 \text{ en el interior de } \Omega \\
        \infty \text{ en el exterior de } \Omega
    \end{cases}
\end{equation}

\vspace{3mm}

Con esta definición del hamiltoniano que describe la dinámica de la partícula entre colisiones podemos ver que la trayectoria seguida será siempre una línea recta, y a una velocidad constante entre colisiones.

\vspace{3mm}

\begin{figure}
    
\end{figure}

\section{Billar clásico}

Situándonos en la teoría clásica vamos a suponer un billar cuyas paredes tienen masa infinita y donde las colisiones producidas son totalmente elásticas. Procedamos a ver como se ve modificada la velocidad de la partícula en el momento de la colisión.

\vspace{3mm}

Al ser una colisión elástica el ángulo de reflexión será igual al ángulo con el que llega la partícula al punto de colisión. En el caso de tener paredes estáticas y una partícula con momento inicial \( \mathbf{p} \), el cambio que se produce en el momento es 

\begin{equation}
    \mathbf{p}' = \mathbf{p} - 2\mathbf{n}(\mathbf{p} \cdot \mathbf{n})
    \label{eq:elastic_col}
\end{equation}

Podemos notar que el cambio del momento se produce únicamente en la dirección normal a la pared dado el producto \( (\mathbf{p} \cdot \mathbf{n}) \), que es nulo cuando son vectores perpendiculares y deja el momento invariante.

\begin{figure}[H]
    \centering
    \includegraphics[scale=1.5]{images/elastic_collision.pdf}
    \caption{Cambio de momento tras una colisión elástica}
    \label{fig:elastic_col}
\end{figure}

\vspace{3mm}

Supongamos ahora que tenemos una partícula con momento \( \mathbf{p} \) y una pared también en movimiento con momento \( \mathbf{P} \) con dirección a la normal. Para obtener el cambio de momento tras la colisión nos trasladamos al sistema de referencia donde la pared está en reposo, así podemos aplicar la ecuación \ref{eq:elastic_col} y volver al sistema de referencia inicial.

\begin{figure}[H]
    \centering
    \includegraphics[scale=1.5]{images/cambio_sistema_referencia.pdf}
    \caption{Posición de la pared y la partícula en los sistemas $S_0$ y $S$}
    \label{fig:elastic_col}
\end{figure}

\vspace{3mm}

El sistema de referencia S sitúa a la pared en su origen y se mueve con velocidad \( \mathbf{w} \) (igual a la del muro). La velocidad relativa de la partícula en ese nuevo sistema es

\begin{equation}
    \mathbf{v}_S = \mathbf{u} - \mathbf{w}
\end{equation}

La nueva velocidad tras colisionar será

\begin{equation}
    \mathbf{v}_S' = \mathbf{v}_S - 2\mathbf{n}(\mathbf{v}_S \cdot \mathbf{n})
\end{equation}

Ahora deshaciendo la transformación de la velocidad y volviendo al sistema \( S_0 \), se obtiene

\begin{align}
    \mathbf{u}' &= \mathbf{v}_S' + \mathbf{w} \\
    &= \mathbf{v}_S - 2\mathbf{n}(\mathbf{v}_S \cdot \mathbf{n}) + \mathbf{w} \\
    &= \mathbf{u} - 2\mathbf{n}\left[(\mathbf{u} - \mathbf{w}) \cdot \mathbf{n}\right]
\end{align}

\vspace{3mm}

\section{Billar relativista}


\end{document}